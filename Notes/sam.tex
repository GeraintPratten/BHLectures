\chapter{SageMath computations} \label{s:sam}

\minitoc

\section{Introduction}


\textsf{SageMath} (\url{https://www.sagemath.org/}) is a modern free,
open-source mathematics software system, which is
based on the Python programming language. It makes use of over 90 open-source packages,
among which are \textsf{Maxima} and \textsf{Pynac} (symbolic calculations),
\textsf{GAP} (group theory),
\textsf{PARI/GP} (number theory), \textsf{Singular} (polynomial computations),
and \textsf{matplotlib} (high quality 2D figures).
\textsf{SageMath} provides a uniform Python interface to all these packages; however,
\textsf{SageMath} is much more than a mere interface: it contains a large and increasing part of
original code (more than 750,000 lines of Python and Cython, involving 5344 classes).
\textsf{SageMath} was created in 2005 by William Stein\index{Stein, W.} \cite{SteinJ05} and since
then its development has been sustained by more than a hundred researchers
(mostly mathematicians). Very good introductory textbooks about \textsf{SageMath} are
\cite{JoyneS14,Zimme13,Bard15}.

The \textsf{SageManifolds} project
provides \textsf{SageMath} with capability for differential geometry and tensor calculus
(\url{https://sagemanifolds.obspm.fr/}), which we are using here to perform
some computations related to the lectures.

There are basically two ways to use \textsf{SageMath}:
\begin{itemize}
\item Install it on your computer, by downloading the sources or a binary version
from \url{https://www.sagemath.org/} (the \textsf{SageManifolds} extensions towards
differential geometry are fully integrated in version 7.5 and higher)
\item Use it online via \textsf{CoCalc}: \url{https://cocalc.com/}
\end{itemize}

\section{SageMath notebooks} \label{s:sam:notebooks}

The SageMath notebooks accompanying these lecture notes are available
at\\

\centerline{
\url{https://luth.obspm.fr/~luthier/gourgoulhon/bh16/sage.html}}

\subsection{The Schwarzschild horizon} \label{s:sam:Schwarz_hor}

This notebook accompanies Chap.~\ref{s:def} in treating the future event horizon of
Schwarzschild spacetime in Eddington-Finkelstein coordinates as an example of null hypersurface:\\[1ex]
{\footnotesize
\url{https://nbviewer.jupyter.org/github/egourgoulhon/BHLectures/blob/master/sage/Schwarzschild_horizon.ipynb}
}

\subsection{Conformal completion of Minkowski spacetime}

This notebook accompanies Chap.~\ref{s:glo}; in particular, it provides
many figures for Sec.~\ref{s:glo:conf_Mink}.\\[1ex]
{\footnotesize
\url{https://nbviewer.jupyter.org/github/egourgoulhon/BHLectures/blob/master/sage/conformal_Minkowski.ipynb}
}

\subsection{Solving Einstein equation: Kottler solution} \label{s:sam:Kottler_solution}

This notebook accompanies Chap.~\ref{s:sch}: it computes the Kottler solution by solving the Einstein
equation for vacuum spherically symmetric spacetimes with a cosmological constant $\Lambda$,
yielding Schwarzschild solution in the special case $\Lambda=0$. \\[1ex]
{\footnotesize
\url{https://nbviewer.jupyter.org/github/egourgoulhon/BHLectures/blob/master/sage/Kottler_solution.ipynb}
}

\subsection{Kretschmann scalar of Schwarzschild spacetime} \label{s:sam:Kretschmann_Schwarz}

This notebook accompanies Chap.~\ref{s:sch}: it computes the Riemann curvature
tensor of Schwarzschild metric and evaluates the Kretschmann scalar as defined
by Eq.~(\ref{e:sch:def_Kretschmann}). \\[1ex]
{\footnotesize
\url{https://nbviewer.jupyter.org/github/sagemanifolds/SageManifolds/blob/master/Notebooks/SM_basic_Schwarzschild.ipynb}
}

\subsection{Radial null geodesics in Schwarzschild spacetime}

This notebook accompanies Chap.~\ref{s:sch}: it provides figures based on
Schwarzschild-Droste coordinates and ingoing Eddington-Finkelstein coordinates.\\[1ex]
{\footnotesize
\url{https://nbviewer.jupyter.org/github/egourgoulhon/BHLectures/blob/master/sage/Schwarz_radial_null_geod.ipynb}
}

\subsection{Radial timelike geodesics in Schwarzschild spacetime} \label{s:sam:ges_radial_free_fall}

This notebook accompanies Chap.~\ref{s:ges}: it provides
figures as well as the computation of
the integral leading to
of Eq.~(\ref{e:ges:sol_t_radial_infall}).\\[1ex]
{\footnotesize
\url{https://nbviewer.jupyter.org/github/egourgoulhon/BHLectures/blob/master/sage/ges_radial_free_fall.ipynb}
}

\subsection{Timelike orbits in Schwarzschild spacetime} \label{s:sam:ges_orbits}
%
This notebook accompanies Chap.~\ref{s:ges}: it provides
figures of timelike orbits in the equatorial plane.\\[1ex]
{\footnotesize
\url{https://nbviewer.jupyter.org/github/egourgoulhon/BHLectures/blob/master/sage/ges_orbits.ipynb}
}

\subsection{Effective potential for null geodesics in Schwarzschild spacetime} \label{s:sam:ges_eff_pot_null}
%
This notebook accompanies Chap.~\ref{s:gis}, providing the plot of the
effective potential $U(r)$.\\[1ex]
{\footnotesize
\url{https://nbviewer.jupyter.org/github/egourgoulhon/BHLectures/blob/master/sage/ges_effective_potential_null.ipynb}
}

\subsection{Null geodesics in Schwarzschild spacetime} \label{s:sam:ges_null_geod}
%
This notebook accompanies Chap.~\ref{s:gis}, computing and plotting various
null geodesics.\\[1ex]
{\footnotesize
\url{https://nbviewer.jupyter.org/github/egourgoulhon/BHLectures/blob/master/sage/ges_null_geod.ipynb}
}

\subsection{Periastron and apoastron of null geodesics in Schwarzschild spacetime} \label{s:sam:ges_null_periastron}
%
This notebook accompanies Chap.~\ref{s:gis}, computing periastrons and apoastrons
along null geodesics.\\[1ex]
{\footnotesize
\url{https://nbviewer.jupyter.org/github/egourgoulhon/BHLectures/blob/master/sage/ges_null_periastron.ipynb}
}

\subsection{Critical null geodesics in Schwarzschild spacetime} \label{s:sam:ges_null_critical_geod}
%
This notebook accompanies Chap.~\ref{s:gis}, plotting critical null geodesics.\\[1ex]
{\footnotesize
\url{https://nbviewer.jupyter.org/github/egourgoulhon/BHLectures/blob/master/sage/ges_null_critical_geod.ipynb}
}

\subsection{Elliptic integrals for null geodesics in Schwarzschild spacetime} \label{s:sam:gis_elliptic_int}
%
This notebook accompanies Chap.~\ref{s:gis}, computing the trace of null
geodesics in the equatorial plane via elliptic integrals.\\[1ex]
{\footnotesize
\url{https://nbviewer.jupyter.org/github/egourgoulhon/BHLectures/blob/master/sage/gis_elliptic_int.ipynb}
}

\subsection{Null geodesics in Schwarzschild spacetime with $b<b_{\rm c}$} \label{s:sam:gis_paramaters_b_lt_bc}
%
This notebook accompanies Chap.~\ref{s:gis}, computing various quantities that
are relevant for null geodesics with an impact parameter lower than the critical one.\\[1ex]
{\footnotesize
\url{https://nbviewer.jupyter.org/github/egourgoulhon/BHLectures/blob/master/sage/gis_paramaters_b_lt_bc.ipynb}
}

\subsection{Multiple images in Schwarzschild spacetime} \label{s:sam:ges_null_images}
%
This notebook accompanies Chap.~\ref{s:gis}, computing null geodesics that
depart in a fixed direction (the observer one).\\[1ex]
{\footnotesize
\url{https://nbviewer.jupyter.org/github/egourgoulhon/BHLectures/blob/master/sage/ges_null_images.ipynb}
}

\subsection{Emission from a point source in Schwarzschild spacetime} \label{s:sam:gis_emission}
%
This notebook accompanies Chap.~\ref{s:gis}, computing quantities related to the
emission by a static observer.\\[1ex]
{\footnotesize
\url{https://nbviewer.jupyter.org/github/egourgoulhon/BHLectures/blob/master/sage/gis_emission.ipynb}
}

\subsection{Images of an accretion disk around a Schwarzschild black hole} \label{s:sam:gis_disk_image}
%
This notebook accompanies Chap.~\ref{s:gis}, computing null geodesics illustrating the formation of images of an accretion disk.\\[1ex]
{\footnotesize
\url{https://nbviewer.jupyter.org/github/egourgoulhon/BHLectures/blob/master/sage/gis_disk_image.ipynb}
}

\subsection{Kruskal-Szekeres coordinates in Schwarzschild spacetime}

This notebook accompanies Chap.~\ref{s:max}: it provides the figures based on
Kruskal-Szekeres coordinates.\\[1ex]
{\footnotesize
\url{https://nbviewer.jupyter.org/github/egourgoulhon/BHLectures/blob/master/sage/Schwarz_Kruskal_Szekeres.ipynb}
}

\subsection{Standard (singular) Carter-Penrose diagram of Schwarzschild spacetime}
\label{s:sam:std_Carter-Penrose}
This notebook accompanies Chap.~\ref{s:max}: it provides the standard
Carter-Penrose diagram shown in Fig.~\ref{f:max:carter-penrose-std}.\\[1ex]
{\footnotesize
\url{https://nbviewer.jupyter.org/github/egourgoulhon/BHLectures/blob/master/sage/Schwarz_conformal_std.ipynb}
}

\subsection{Regular Carter-Penrose diagram of Schwarzschild spacetime}
\label{s:sam:reg_Carter-Penrose}
This notebook accompanies Chap.~\ref{s:max}: it provides the regular
Carter-Penrose diagram shown in Fig.~\ref{f:max:carter-penrose-FN}.\\[1ex]
{\footnotesize
\url{https://nbviewer.jupyter.org/github/egourgoulhon/BHLectures/blob/master/sage/Schwarz_conformal.ipynb}
}

\subsection{Einstein-Rosen bridge in Schwarzschild spacetime}
\label{s:sam:Einstein-Rosen}
This notebook accompanies Chap.~\ref{s:max}: it provides the
isometric embedding diagrams shown in Figs.~\ref{f:max:flamm_paraboloid}
to \ref{f:max:flamm_paraboloid_far}, as well as the associated
Kruskal diagram of Fig.~\ref{f:max:constant_T_slices}.\\[1ex]
{\footnotesize
\url{https://nbviewer.jupyter.org/github/egourgoulhon/BHLectures/blob/master/sage/Einstein-Rosen_bridge.ipynb}
}


\subsection{Kerr metric as a solution of Einstein equation} \label{s:sam:Kerr_solution}

This notebook accompanies Chap.~\ref{s:ker}: the Kerr metric, expressed in Boyer-Lindquist
coordinates, is shown to be a solution of the vacuum Einstein equation. Moreover, the Kretschmann scalar is computed.\\[1ex]
{\footnotesize
\url{https://nbviewer.jupyter.org/github/egourgoulhon/BHLectures/blob/master/sage/Kerr_solution.ipynb}
}

\subsection{Kerr spacetime in 3+1 Kerr coordinates} \label{s:sam:Kerr_Kerr_coord}

This notebook accompanies Chap.~\ref{s:ker}: the Kerr metric is expressed in 3+1 Kerr coordinates, the vacuum Einstein equation is checked, the outgoing and ingoing principal null geodesics are considered and the black hole surface gravity is computed.\\[1ex]
{\footnotesize
\url{https://nbviewer.jupyter.org/github/egourgoulhon/BHLectures/blob/master/sage/Kerr_in_Kerr_coord.ipynb}
}

\subsection{Kerr-Schild coordinates on Kerr spacetime} \label{s:sam:Kerr_Schild}

This notebook accompanies Appendix.~\ref{s:ksm}: the Kerr metric is expressed in
a Kerr-Schild form and Kerr-Schild coordinates are introduced.\\[1ex]
{\footnotesize
\url{https://nbviewer.jupyter.org/github/egourgoulhon/BHLectures/blob/master/sage/Kerr_Schild.ipynb}
}

\subsection{Walker-Penrose Killing tensor on Kerr spacetime} \label{s:sam:Kerr_Killing_tensor}

This notebook accompanies Appendix.~\ref{s:gek}; it shows that the
tensor $\w{K}$ defined by Eq.~(\ref{e:gek:def_K}) is actually a Killing tensor.\\[1ex]
{\footnotesize
\url{https://nbviewer.jupyter.org/github/egourgoulhon/BHLectures/blob/master/sage/Kerr_Killing_tensor.ipynb}
}


\subsection{Lemaître-Tolman equations} \label{s:sam:Lemaitre-Tolman}

This notebook accompanies Chap.~\ref{s:lem}: it provides the derivation
of the Lemaître-Tolman equations from the Einstein equation expressed in
Lemaître synchronous coordinates.\\[1ex]
{\footnotesize
\url{https://nbviewer.jupyter.org/github/egourgoulhon/BHLectures/blob/master/sage/Lemaitre_Tolman.ipynb}
}

\subsection{Trapping horizon in Vaidya spacetime} \label{s:sam:Vaidya_trapping}

This notebook accompanies Chap.~\ref{s:loc}: the Vaidya metric is expressed in  Eddington-Finkelstein coordinates, the Einstein equation is checked, the outgoing and ingoing radial null geodesics are computed and the trapping horizon and the event
horizon are drawn in a spacetime diagram. \\[1ex]
{\footnotesize
\url{https://nbviewer.jupyter.org/github/egourgoulhon/BHLectures/blob/master/sage/Vaidya.ipynb}
}
