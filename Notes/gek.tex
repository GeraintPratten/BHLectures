\chapter{Geodesics in Kerr spacetime}
\label{s:gek}
\index{geodesic!in Kerr spacetime}

\minitoc

\section{Introduction}

\section{Equations of geodesic motion}

\subsection{Introduction}

In all this chapter, we are concerned by the motion of a particle
$\mathscr{P}$ in Kerr spacetime $(\M,\w{g})$, assuming that $\mathscr{P}$
feels only gravity, as described by $\w{g}$ (freely falling particle).
The worldline $\Li$ of $\mathscr{P}$
is then necessarily a geodesic\footnote{The definition and basic properties of geodesics
are recalled in Appendix~\ref{s:geo}; see also Sec.~\ref{s:fra:geod_motion}.} of
$(\M,\w{g})$. It is a timelike geodesic if $\mathscr{P}$ is a massive particle
and a null geodesic if $\mathscr{P}$ is massless (e.g. a photon).

In a given coordinate system $(x^\alpha)$, the geodesic $\Li$ is described
by a system of the form\footnote{In Appendices~\ref{s:bas} and \ref{s:geo},
we use a different notation for the function $X^\alpha(\lambda)$ defining
$\Li$ and the coordinate $x^\alpha$ (cf. Secs.~\ref{s:bas:vectors} and
\ref{s:geo:geodesic_eq}). Following standard usage in the physics literature,
we shall not do this here.}
$x^\alpha = x^\alpha(\lambda)$, where $\lambda$ is an
affine parameter\footnote{See Sec.~\ref{s:geo:def} for the definition
of an affine parameter along a geodesic.} along $\Li$. We choose $\lambda$ to be the affine parameter
associated with $\mathscr{P}$'s 4-momentum $\w{p}$ [cf. Eq.~(\ref{e:fra:p_dxdl})]. We have
then $\D x^\alpha / \D\lambda = p^\alpha$.
The curve $\Li$ is a geodesic iff the functions $x^\alpha(\lambda)$ obey
the geodesic equation [Eq.~(\ref{e:geo:eq_geod}) in Appendix~\ref{s:geo}]:
\be \label{e:gek:eq_geod}
    \frac{\D^2 x^\alpha}{\D\lambda^2} + \Gamma^\alpha_{\ \, \mu \nu}
    \frac{\D x^\mu}{\D\lambda} \frac{\D x^\nu}{\D\lambda} = 0 ,
\ee
where the $\Gamma^\alpha_{\ \, \mu \nu}$'s are the Christoffel symbols\index{Christoffel symbols} of the metric $\w{g}$
with respect to the coordinates $(x^\alpha)$, as given by Eq.~(\ref{e:bas:Christoffel}).
Equation~(\ref{e:gek:eq_geod}) is actually a system of four coupled second-order
differential equations, that are non-linear. We are going to see that in
order to compute the geodesics of Kerr spacetime, it is not necessary to
solve this system, since, as in the Schwarzschild case studied in Chap.~\ref{s:ges}, we have
enough first integrals of Eq.~(\ref{e:gek:eq_geod}) to reduce
the problem to four first-order equations
of the type $\D x^\alpha/\D\lambda = F^\alpha(x^0, x^1, x^2, x^3)$.

Three integrals of motion are similar to those in the Schwarzschild case.
One is the particle's mass $\mu$ (Sec.~\ref{s:gek:mass_int_motion} below)
and the two other ones are
the conserved energy $E$ and the conserved angular momentum
$L$ (Sec.~\ref{s:gek:int_motion_sym}), which arise from the common symmetries
of Kerr and Schwarzschild spacetimes: stationarity (outside the ergosphere) and axisymmetry.
In the Schwarzschild case, the fourth integral of motion was provided by
spherical symmetry, which constrained all geodesics to be planar, so that
a suitable choice of
coordinates $(t,r,\th,\ph)$ makes a given geodesic confined to the
hyperplane $\th=\pi/2$, yielding the first integral $p^\th=0$ [Eq.~(\ref{e:ges:pth_zero})].
The Kerr spacetime with $a\not=0$ being not spherically symmetric, we loose
this property here. Fortunaly there exists another
integral of motion: the Carter constant; it arises from a remarkable property
of Kerr spacetime: the existence of a non-trivial Killing tensor (Sec.~\ref{s:gek:Carter_const}).
With the particle mass $\mu$, this makes a total of four integral of motions,
which renders the problem integrable (Sec.~\ref{s:gek:first_order_system}).


\subsection{Integrals of motion from spacetime symmetries} \label{s:gek:int_motion_sym}

As for Schwarzschild spacetime (cf. Sec.~\ref{s:ges:fiom}), we have the following
property.
\begin{greybox}
The Killing vectors $\w{\xi}$ and $\w{\eta}$ of Kerr spacetime,
associated respectively with
stationarity and axisymmetry [cf. Eq.~(\ref{e:ker:def_xi_eta})],
give birth to two conserved quantities
along the geodesic $\Li$:
\begin{subequations}
\label{e:gek:def_E_L}
\begin{align}
& \encadre{E := - \w{\xi}\cdot \w{p} = - \w{g}(\w{\xi},\w{p}) } \label{e:gek:def_E} \\
& \encadre{L := \w{\eta}\cdot \w{p} = \w{g}(\w{\eta},\w{p}) } , \label{e:gek:def_L}
\end{align}
\end{subequations}
where $\w{p}$ is the 4-momentum of particle $\mathscr{P}$ (cf. Sec.~\ref{s:fra:worldlines}).
For the same reasons as in Sec.~\ref{s:ges:fiom}, $E$ is called
the \defin{conserved energy}\index{conserved!energy}\index{energy!conserved --}
or \defin{energy at infinity}\index{energy!at infinity} of $\mathscr{P}$,
while $L$ is called $\mathscr{P}$'s \defin{conserved angular momentum}\index{conserved!angular momentum}\index{angular momentum!conserved --}
or \defin{angular momentum at infinity}\index{angular momentum!at infinity}
of $\mathscr{P}$.
\end{greybox}
\begin{remark}
Asymptotically, the scalar $L$ is only the
component along the rotation axis of $\mathscr{P}$'s angular momentum vector $\w{L}$
as measured by an inertial observer at rest with respect to the black hole.
For this reason, it is sometimes denoted by $L_z$, instead of merely $L$.
\end{remark}

In coordinates $(t,r,\th,\ph)$ adapted to the spacetime symmetries,
i.e. coordinates such that $\w{\xi} = \wpar_t$ and $\w{\eta}=\wpar_\ph$,
for instance Boyer-Lindquist coordinates (Sec.~\ref{s:ker:expr_BL}),
Kerr coordinates (Sec.~\ref{s:ker:Kerr_coord}) or 3+1 Kerr coordinates
(Sec.~\ref{s:ker:3p1_Kerr_coord}), one can rewrite
(\ref{e:gek:def_E_L})
in terms of the components $p_t = g_{t\mu} \, p^\mu$ and $p_\ph = g_{\ph\mu} \, p^\mu$
of the 1-form $\uu{p}$ associated to $\w{p}$ by metric duality:
\begin{subequations}
\label{e:gek:E_pt_L_pph}
\begin{align}
& E = - p_t \\
& L = p_\ph
\end{align}
\end{subequations}
Indeed, in such a coordinate system, $\xi^\mu =  \delta^\mu_{\ \, t}$
and $\eta^\mu = \delta^\mu_{\ \, \ph}$, so that $E = -g_{\mu\nu} \, \xi^\mu p^\nu = -g_{t\nu} \, p^\nu = -p_t$
and $L = g_{\mu\nu} \, \eta^\mu p^\nu = g_{\ph\nu} \, p^\nu = p_\ph$.

In what follows, we are going to use Boyer-Lindquist coordinates
$(x^\alpha)=(t,r,\th,\ph)$
as introduced in Sec.~\ref{s:ker:expr_BL}.
Given the components (\ref{e:ker:metric_BL}) of the metric tensor $\w{g}$
in these coordinates, evaluating $E$ and $L$
via $E = - g_{t\mu} \, p^\mu$ and $L = g_{\ph\mu} \, p^\mu$ yields
\be \label{e:gek:E_first_int}
    E = \left( 1 - \frac{2 m r}{\rho^2} \right)\,  p^t
        + \frac{2 a m r \sin^2\th}{\rho^2}\,  p^\ph  .
\ee
\be \label{e:gek:L_first_int}
    L = - \frac{2 a m r \sin^2\th}{\rho^2} \, p^t
        + \left( r^2 + a^2 + \frac{2 a^2 m r \sin^2\th}{\rho^2} \right)
    \sin^2\th \,  p^\ph ,
\ee
where $\rho^2 := r^2 + a^2 \cos^2\th$ [Eq.~(\ref{e:ker:def_rho2})].

Let us recall that the components $(p^\alpha)$ of the 4-momentum are
related to the equation $x^\alpha = x^\alpha(\lambda)$ of the geodesic $\Li$
in terms of the affine parameter $\lambda$ by $p^\alpha = \D x^\alpha / \D\lambda$
[cf. Eq.~(\ref{e:fra:p_dxdl})], i.e.
\be \label{e:gek:pa_der_xa}
    p^t = \derd{t}{\lambda},\quad
    p^r = \derd{r}{\lambda},\quad
    p^\th = \derd{\th}{\lambda},\quad
    p^\ph = \derd{\ph}{\lambda} .
\ee

\subsection{Mass as an integral of motion} \label{s:gek:mass_int_motion}

The mass $\mu$ of particle $\mathscr{P}$ is related to the scalar square of
the 4-momentum vector $\w{p}$ via Eq.~(\ref{e:fra:def_mass}):
\be \label{e:gek:mu_gpp}
    \mu^2 = - \w{g}(\w{p}, \w{p}) .
\ee
The fact that $\Li$ is a geodesic implies that $\mu$ is a constant along
$\Li$ (cf. Eq.~(\ref{e:geo:vv_const}) in Appendix~\ref{s:geo}). It therefore
provides a third integral of motion, after $E$ and $L$.

It is convenient to express (\ref{e:gek:mu_gpp}) in terms of the inverse metric
in order to let appear $p_t = -E$ and $p_\ph = L$:
\[
    \mu^2 = - g^{\mu\nu} p_\mu p_\nu .
\]

Given the components (\ref{e:ker:inv_met_BL}) of the inverse metric
in Boyer-Lindquist coordinates, we get
\bea
    \mu^2 & = & \frac{1}{\Delta}
    \left( r^2 + a^2 + \frac{2 a^2 m r \sin^2\th}{\rho^2} \right) E^2
    -\frac{4 a m r}{\rho^2 \Delta} E L
    - \frac{1}{\Delta}\left(1 - \frac{2 m r}{\rho^2} \right) \frac{L^2}{\sin^2\th}
    \nonumber \\
   &  & - \frac{\rho^2}{\Delta} \left( p^r \right) ^2
    - \rho^2 \left( p^\theta \right) ^2 ,   \label{e:gek:mu2_first_int}
\eea
where $\Delta := r^2 - 2 m r + a^2$ [Eq.~(\ref{e:ker:def_Delta})].
Note that we have expressed $p_r$ and $p_\th$ in terms of $p^r$ and $p^\th$
thanks to the relations $p_r = g_{r\mu} p^\mu$ and $p_\th = g_{\th\mu} p^\mu$,
which are very simple for the Boyer-Lindquist components (\ref{e:ker:metric_BL})
of $\w{g}$:
\be \label{e:gek:p_r_p_th_cov_con}
    p_r = \frac{\rho^2}{\Delta} \, p^r
    \qquad\mbox{and}\qquad
    p_\th = \rho^2 \, p^\th .
\ee

\subsection{The fourth integral of motion: Carter constant} \label{s:gek:Carter_const}

It turns out that the Kerr spacetime is endowed with a non-trivial
Killing tensor of valence 2: the
\defin{Walker-Penrose Killing tensor}\index{Killing!tensor!Walker-Penrose --}\index{Walker-Penrose Killing tensor} $\w{K}$, which is the symmetric tensor of type
$(0,2)$ defined by
\be \label{e:gek:def_K}
    \encadre{ \w{K} := (r^2 + a^2) \left( \uu{k} \otimes \uu{\el} + \uu{\el} \otimes \uu{k} \right) + r^2 \w{g} } ,
\ee
where $\uu{\el}$ and $\uu{k}$ are the 1-forms associated by metric duality
to the null vector fields $\w{k}$ and $\wl$ tangent to the principal null
geodesics introduced in Sec.~\ref{s:ker:principal_geod}. In index notation,
Eq.~(\ref{e:gek:def_K}) writes
\be
    K_{\alpha\beta} = (r^2 + a^2) \left(k_\alpha \el_\beta + \el_\alpha k_\beta \right)
        + r^2 g_{\alpha\beta} .
\ee
$\w{K}$ is called a \defin{Killing tensor}\index{Killing!tensor} because its symmetrized covariant derivative vanishes identically:
\be \label{e:gek:Killing_eq_K}
    \encadre{ \nabla_{(\alpha} K_{\beta\gamma)} = 0 }.
\ee
This property can be seen as a generalization of the Killing equation
(\ref{e:neh:Killing_equation}) to tensors of valence 2.
That the tensor $\w{K}$ defined by (\ref{e:gek:def_K}) obeys the Killing
identity (\ref{e:gek:Killing_eq_K}) is shown in the notebook~\ref{s:sam:Kerr_Killing_tensor}.

Killing tensors are discussed in Sec.~\ref{e:geo:Killing_tensor} of Appendix~\ref{s:geo}.
It is shown there that the Killing identity (\ref{e:gek:Killing_eq_K}) implies that
the following quantity is constant along any geodesic $\Li$ [cf. Eq.~(\ref{e:geo:Kvv_const})]:
\be
    \encadre{ \mathscr{K} := \w{K}(\w{p}, \w{p}) = K_{\mu\nu} p^\mu p^\nu } .
\ee
$\mathscr{K}$ is named \defin{Carter constant}\index{Carter!constant}.
From the definition (\ref{e:gek:def_K}) of $\w{K}$, we have
\be \label{e:gek:Kcarter_prov}
    \mathscr{K} = 2(r^2 + a^2) \langle \uu{k}, \w{p} \rangle
        \langle \uu{\el}, \w{p} \rangle + r^2 \w{g}(\w{p}, \w{p}).
\ee
Now by Eq.~(\ref{e:gek:mu_gpp}), $\w{g}(\w{p}, \w{p}) = - \mu^2$. Besides, we
have
\[
  \langle \uu{k}, \w{p} \rangle = k_\mu p^\mu = k^\mu p_\mu .
\]
The last form allows us to let appear the constants of motion $p_t = -E$ and
$p_\ph = L$ [Eq.~(\ref{e:gek:E_pt_L_pph})]. Using it with the components
of $\w{k}$ as given by Eq.~(\ref{e:ker:k_BL}), we get
\[
    \langle \uu{k}, \w{p} \rangle = - \frac{r^2 + a^2}{\Delta} E
    - p_r + \frac{a}{\Delta} L .
\]
Similarly, from the components (\ref{e:ker:ell_BL}) of $\wl$, we obtain
\[
     \langle \uu{\el}, \w{p} \rangle = - \frac{1}{2} E + \frac{\Delta}{2(r^2 + a^2)} p_r
     + \frac{a}{2(r^2 + a^2)} L .
\]
Accordingly, Eq.~(\ref{e:gek:Kcarter_prov}) becomes
\[
    \mathscr{K} = \frac{1}{\Delta} \left[ (r^2 + a^2) E + \Delta p_r - a L \right]
      \left[ (r^2 + a^2) E - \Delta p_r - a L \right] - r^2 \mu^2 ,
\]
which can be rewritten as
\be \label{e:gek:Kcarter_first_int}
   \encadre{ \mathscr{K} = \frac{1}{\Delta} \left[ \left( (r^2 + a^2) E - a L \right)^2
        - \rho^4 (p^r)^2 \right] - r^2 \mu^2 }.
\ee
Note that we have expressed $p_r$ in terms of $p^r$ via Eq.~(\ref{e:gek:p_r_p_th_cov_con}).

We can get some physical interpretation of the Carter constant from the above
expression, in the case where the particle $\mathscr{P}$ following the
geodesic $\Li$ visits the asymptotic region $r\to+\infty$. Indeed, given
that $\Delta := r^2 - 2m r + a^2 \sim r^2$  and $\rho^4 := (r^2 + a^2\cos^2\th)^2 \sim r^4$
as $r\to+\infty$, we deduce from Eq.~(\ref{e:gek:Kcarter_first_int})
that
\be \label{e:gek:Kcarter_r_inf}
    \mathscr{K} \underset{r\to + \infty}{\sim} r^2 \left[ E^2 - \mu^2 - (p^r)^2 \right] .
\ee
Then let $\Obs$ be the asymptotic inertial observer at rest with respect to the
black hole, i.e. the observer with $r\gg m$ and whose 4-velocity
coincides with the Killing vector
$\w{\xi} = \wpar_t$. $E$ is then $\mathscr{P}$'s energy as measured by $\Obs$
[cf. Eq.~(\ref{e:fra:E_obs}) and (\ref{e:gek:def_E})], so that, according to
Einstein's formula (\ref{e:fra:E2_m2_P2}), $E^2 - \mu^2 = \w{P}\cdot\w{P}$, where
$\w{P}$ is $\mathscr{P}$'s linear momentum as measured by $\Obs$. Given
that asymptotically, $p^r\sim P^r$ [cf. Eq.~(\ref{e:fra:p_E_P})],
Eq.~(\ref{e:gek:Kcarter_r_inf}) becomes
\[
    \mathscr{K} \underset{r\to + \infty}{\sim} r^2 \left[ \w{P}\cdot\w{P} - (P^r)^2 \right]
    = r^2 \left[ (P^{(\th)})^2 + (P^{(\ph)})^2 \right],
\]
where $P^{(\th)} = r P^\th \sim r p^\th$ and $P^{(\ph)} = r\sin\th\, P^\ph \sim r\sin\th\, p^\ph$
are the angular components of $\w{P}$ in the
orthonormal basis $(\w{e}_{(r)}, \w{e}_{(\th)}, \w{e}_{(\ph)}) := (\wpar_r, r^{-1} \wpar_\th,
(r\sin\th)^{-1}\wpar_\ph)$.
Now the total angular momentum of $\mathscr{P}$ measured by $\Obs$ is
\[
    \w{L}_{\rm tot} = \w{r} \times \w{P} = - r P^{(\ph)} \w{e}_{(\th)}
        + r P^{(\th)} \w{e}_{(\ph)} .
\]
Hence we may conclude that asymptotically, the Carter constant coincides with
the square of $\w{P}$'s angular momentum as measured by the inertial observer $\Obs$:
\be \label{e:gek:Kcarter_asymptot}
    \mathscr{K} \underset{r\to + \infty}{\sim} \w{L}_{\rm tot} \cdot \w{L}_{\rm tot} .
\ee

\subsection{First order equations of motion} \label{s:gek:first_order_system}

We have thus obtained four first integrals of the geodesic equation
(\ref{e:gek:eq_geod}):
$E$ [Eq.~(\ref{e:gek:E_first_int})], $L$ [Eq.~(\ref{e:gek:L_first_int})],
$\mu^2$ [Eq.~(\ref{e:gek:mu2_first_int})] and $\mathscr{K}$
[Eq.~(\ref{e:gek:Kcarter_first_int})]. In the expressions of each of these integrals,
$p^\alpha$ has to be thought of as the first order derivative $\D x^\alpha/\D\lambda$
[Eq.~(\ref{e:gek:pa_der_xa})].
Two first integrals, namely $E$ and $L$, are linear in the $p^\alpha$'s, while
the two others, namely $\mu^2$ and $\mathscr{K}$, are quadratic.
Furthermore, Eqs.~(\ref{e:gek:E_first_int}) and (\ref{e:gek:L_first_int})
constitute a decoupled subsystem for $(p^t, p^\ph)$, which can easily be solved\footnote{An
intermediate step is combining  Eqs.~(\ref{e:gek:E_first_int}) and (\ref{e:gek:L_first_int})
to get $a E - L /\sin^2\th = a p^t - (r^2 + a^2) p^\ph$
and $(r^2 + a^2) E - a L = \Delta(p^t - a\sin^2\th p^\ph)$.},
yielding
\be \label{e:gek:rho2_pt}
    \rho^2 p^t = \frac{r^2 + a^2}{\Delta} [(r^2 + a^2) E - a L] + a ( L - a E \sin^2\th )
\ee
\be  \label{e:gek:rho2_pph}
    \rho^2 p^\ph = \frac{L}{\sin^2\th} - a E
    + \frac{a}{\Delta}\left[(r^2 + a^2) E - a L\right] .
\ee
Besides, Eq.~(\ref{e:gek:Kcarter_first_int}) involves only $p^r$ and can
be recast as
\be \label{e:gek:p_r_R}
    \rho^2 p^r = \eps_r \sqrt{ R(r) } ,
\ee
where $\eps_r := \operatorname{sgn} p^r = \pm 1$ and
$R(r)$ is the following 4th order polynomial of $r$:
\be \label{e:gek:def_R}
   \encadre{ R(r) := \left[ (r^2 + a^2) E - a L \right]^ 2 - \Delta (r^2 \mu^2 + \mathscr{K}) }.
\ee
In the above expression, let us recall that $\Delta$ is the function of $r$
given by Eq.~(\ref{e:ker:def_Delta}): $\Delta := r^2 - 2 m r + a^2$. All other
quantities are constant.

Finally, if we substitute $p^r$ by the value given by Eqs.~(\ref{e:gek:p_r_R})-(\ref{e:gek:def_R}) in the mass first integral (\ref{e:gek:mu2_first_int}), we get, after simplification,
\be \label{e:gek:p_th_Th}
    \rho^2 p^\th = \eps_\th \sqrt{\Theta(\th)} ,
\ee
where $\eps_\th := \operatorname{sgn} p^\th = \pm 1$ and
\be \label{e:gek:def_Theta}
   \encadre{ \Theta(\th) := \mathscr{K} - \left(\frac{L}{\sin\th} - a E \sin\th \right)^2
        - \mu^2 a^2 \cos^2\th } .
\ee
The following constant is often used instead of $\mathscr{K}$:
\be \label{e:gek:def_Q}
    \encadre{Q := \mathscr{K} - (L - a E)^2 }
\ee
Thanks to it, we may rewrite (\ref{e:gek:def_Theta}) as
\be \label{e:gek:Theta_Q}
    \encadre{ \Theta(\th) = Q + \cos^2\th \left[ a^2 (E^2 - \mu^2)
    - \frac{L^2}{\sin^2\th} \right] } .
\ee
In the literature, $Q$ is often called \emph{Carter's constant} as well. A difference
with the original Carter constant $\mathscr{K}$ is that the latter is always non-negative,
as Eq.~(\ref{e:gek:def_Theta}) shows (given that $\Theta(\th) \geq 0$ by virtue of
Eq.~(\ref{e:gek:p_th_Th})):
\be
    \mathscr{K} \geq 0 .
\ee
If the particle $\mathscr{P}$ reaches the asymptotic region $r\gg m$, we deduce from
Eqs.~(\ref{e:gek:def_Q}) and (\ref{e:gek:Kcarter_asymptot}) the following approximate
value of $Q$:
\be
   Q \underset{r\to + \infty}{\sim} \w{L}_{\rm tot} \cdot \w{L}_{\rm tot}
    - L^2 + a (2 L - a E^2) .
\ee
Hence, if $a=0$, $Q$ can be interpreted as the square of
the part of $\mathscr{P}$'s angular momentum (measured by the asymptotic inertial
observer) that is not in $L$. Using
notations of Sec.~\ref{s:gek:Carter_const}, we have $L = L_{\rm tot}^{(\ph)}$,
so that $ \w{L}_{\rm tot} \cdot \w{L}_{\rm tot} - L^2 = (L_{\rm tot}^{(\th)})^2$.

In view of the relation (\ref{e:gek:pa_der_xa}) between the $p^\alpha$'s
and the derivatives of the functions $x^\alpha(\lambda)$, we may
collect Eqs.~(\ref{e:gek:rho2_pt}), (\ref{e:gek:rho2_pph}), (\ref{e:gek:p_r_R})
and (\ref{e:gek:def_Theta}) as the first-order system
\begin{subequations}
\label{e:gek:eom_first_order}
\begin{align}
& \encadre{ \rho^2 \derd{t}{\lambda} = \frac{r^2 + a^2}{\Delta} [(r^2 + a^2) E - a L] + a ( L - a E \sin^2\th ) } \label{e:gek:dtdl} \\
& \encadre{ \rho^2 \derd{r}{\lambda} = \eps_r \sqrt{ R(r) } } \label{e:gek:drdl}\\
& \encadre{ \rho^2 \derd{\th}{\lambda} = \eps_\th \sqrt{\Theta(\th)} } \label{e:gek:dthdl}\\
& \encadre{ \rho^2 \derd{\ph}{\lambda}  = \frac{L}{\sin^2\th} - a E
    + \frac{a}{\Delta}\left[(r^2 + a^2) E - a L\right] } . \label{e:gek:dphdl}
\end{align}
\end{subequations}
The functions $R(r)$ and $\Theta(\th)$ are defined by Eq.~(\ref{e:gek:def_R})
and Eq.~(\ref{e:gek:def_Theta}) or (\ref{e:gek:Theta_Q}). They must obey
\be \label{e:gek:R_Theta_positive}
    R(r) \geq 0 \qquad\mbox{and}\qquad \Theta(\th) \geq 0 .
\ee
Note that besides the constants of motion $E$, $L$, $\mu$ and $\mathscr{K}$
or $Q$, $R(r)$ depends on $a$ and $m$ (via $\Delta$), while $\Theta(\th)$
depends on $a$ only.

In the system (\ref{e:gek:eom_first_order}), $\eps_r$
is $+1$ (resp. $-1$) in the parts of the geodesic $\Li$ where $r$ increases
(resp. decreases) with $\lambda$. The factor $\eps_r$ changes sign at points where $\D r/\D\lambda = 0$. Such points
are called \defin{radial turning points}\index{radial!turning point}\index{turning point!radial --}. According to Eq.~(\ref{e:gek:drdl}), they correspond to the roots of the polynomial $R(r)$:
\be
    \derd{r}{\lambda} = 0 \iff R(r) = 0 .
\ee
Similarly, $\eps_\th$
is $+1$ (resp. $-1$) in the parts of $\Li$ where $\th$ increases
(resp. decreases) with $\lambda$ and $\eps_\th$ changes sign at the
\defin{latitudinal turning points}\index{latitudinal!turning point}\index{turning point!latitudinal --}, where $\D\th/\D\lambda = 0$. According to Eq.~(\ref{e:gek:dthdl}),
these turning points are given by the roots of the function $\Theta(\th)$:
\be
    \derd{\th}{\lambda} = 0 \iff \Theta(\th) = 0 .
\ee


\subsection{Equations of motion in terms of Mino time}

In view of the right-hand sides of the system (\ref{e:gek:eom_first_order}),
it is quite natural to introduce along the geodesic $\Li$ the new
parameter $\lambda'$ defined by
\be \label{e:gek:def_Mino_time}
    \D\lambda' = \frac{\D\lambda}{\rho^2}
     = \frac{\D\lambda}{r(\lambda)^2 + a^2 \cos^2 \th(\lambda)} .
\ee
Since $\rho^2$ never vanishes on the spacetime manifold $\M$ (by construction
of the latter, cf. Eq.~(\ref{e:ker:def_M_Kerr_spacetime})), the above relation
leads to a well-defined parameter along $\Li$. Moreover, since $\rho^2>0$, this
parameter increases towards the future, as $\lambda$. A difference between
the two parametrizations is that $\lambda'$ is not in general\footnote{The only
exception is for a circurlar orbit at constant $\theta$, since then $\rho^2$
is constant and Eq.~(\ref{e:gek:def_Mino_time}) defines an affine transformation
between $\lambda$ and $\lambda'$.} an affine parameter
of $\Li$, contrary to $\lambda$.
The parameter $\lambda'$ is called \defin{Mino time}\index{Mino!time}\index{time!Mino --}
\cite{Mino03}.

In terms of Mino time, the system (\ref{e:gek:eom_first_order}) becomes
\begin{subequations}
\label{e:gek:eom_Mino}
\begin{align}
& \encadre{ \derd{t}{\lambda'} = \frac{r^2 + a^2}{\Delta} [(r^2 + a^2) E - a L] + a ( L - a E \sin^2\th ) } \label{e:gek:dtdl_Mino} \\
& \encadre{ \derd{r}{\lambda'} = \eps_r \sqrt{ R(r) } } \label{e:gek:drdl_Mino}\\
& \encadre{ \derd{\th}{\lambda'} = \eps_\th \sqrt{\Theta(\th)} } \label{e:gek:dthdl_Mino}\\
& \encadre{ \derd{\ph}{\lambda'}  = \frac{L}{\sin^2\th} - a E
    + \frac{a}{\Delta}\left[(r^2 + a^2) E - a L\right] } . \label{e:gek:dphdl_Mino}
\end{align}
\end{subequations}
It is remarkable that Eqs.~(\ref{e:gek:drdl_Mino}) and (\ref{e:gek:dthdl_Mino})
are two fully decoupled equations: Eq.~(\ref{e:gek:drdl_Mino}) is an ordinary
differential equation for the function $r(\lambda')$, while Eq.~(\ref{e:gek:dthdl_Mino})
is an ordinary differential equation for the function $\th(\lambda')$. This was
not the case for Eqs.~(\ref{e:gek:drdl}) and (\ref{e:gek:dthdl}) since $\rho^2$ is
a function of both $r$ and $\th$.
Equation~(\ref{e:gek:drdl_Mino}) can be integrated by the method of separation
of variables. On a portion of $\Li$ without any radial turning point, this yields
\be
    \lambda' - \lambda'_0 = \eps_r \int_{r_0}^r \frac{\D \bar{r}}{\sqrt{R(\bar{r})}} ,
\ee
with $r_0 := r(\lambda'_0)$.
If there
are $M\geq 1$ radial turning points $r_1,\ldots, r_M$ between $\lambda'_0$ and $\lambda'$,
the formula becomes
\be \label{e:gek:Minotime_r_sum}
    \lambda' - \lambda'_0 = \eps_r^{(0)} \int_{r_0}^{r_1} \frac{\D \bar{r}}{\sqrt{R(\bar{r})}}
    + \sum_{i=1}^{M-1} \eps_{r}^{(i)}  \int_{r_i}^{r_{i+1}} \frac{\D \bar{r}}{\sqrt{R(\bar{r})}}
    + \eps_{r}^{(M)} \int_{r_M}^r \frac{\D \bar{r}}{\sqrt{R(\bar{r})}} .
\ee
where $\eps_r^{(i)}$ is the value of $\eps_r$ between $r_i$ and $r_{i+1}$
for $0\leq i \leq M$,
with the convention $r_{M+1}:=r$. Note that each term is the above sum
is positive, since $\eps_r^{(i)} = -1$ compensates for $r_{i+1} < r_i$.
For instance, if there is a single turning point $r_1$ between
between $\lambda'_0$ and $\lambda'$, Eq.~(\ref{e:gek:Minotime_r_sum}) gives
\[
   \lambda' - \lambda'_0 = \begin{cases}
  \displaystyle \int_{r_0}^{r_1} \frac{\D \bar{r}}{\sqrt{R(\bar{r})}}
    + \int_{r}^{r_1} \frac{\D \bar{r}}{\sqrt{R(\bar{r})}} & \mbox{if} \ r_1 > r_0 \\[2ex]
  \displaystyle \int_{r_1}^{r_0} \frac{\D \bar{r}}{\sqrt{R(\bar{r})}}
    + \int_{r_1}^{r} \frac{\D \bar{r}}{\sqrt{R(\bar{r})}} & \mbox{if} \ r_1 < r_0 ,
    \end{cases}
\]
where all the integrals in the right-hand side are positive.
Moreover, despite $R(r_i) = 0$ for $1\leq r_i \leq M$
(by definition of a turning point), all the integrals involved in
Eq.~(\ref{e:gek:Minotime_r_sum}) are finite provided that the $r_i$'s are \emph{single}
zeros of $R$ (same argument as in Sec.~\ref{s:gis:deflect_winding}). If $r_i$
happens to be a double zero of $R$, then the integral having $r_i$ as a bound
is divergent.

The right-hand side of Eq.~(\ref{e:gek:Minotime_r_sum}) is actually a path
integral and is often abriged by means of a slash notation:
\be
    \lambda' - \lambda'_0 = \dashint_{r_0}^r \frac{\eps_r \, \D \bar{r}}{\sqrt{R(\bar{r})}} .
\ee

Similarly, Eq.~(\ref{e:gek:dthdl_Mino}) can be integrated as
\bea
    \lambda' - \lambda'_0  & = & \dashint_{\th_0}^\th \frac{\eps_\th \, \D \bar{\th}}{\sqrt{Q(\bar{\th})}} \nonumber \\
    & = & \eps_\th^{(0)} \int_{\th_0}^{\th_1} \frac{\D \bar{\th}}{\sqrt{Q(\bar{\th})}}
        + \sum_{j=1}^{N-1} \eps_{\th}^{(j)}
        \int_{\th_j}^{\th_{j+1}} \frac{\D \bar{\th}}{\sqrt{Q(\bar{\th})}}
        + \eps_\th^{(N)}  \int_{\th_N}^{\th} \frac{\D \bar{\th}}{\sqrt{Q(\bar{\th})}} ,
\eea
where $\th_0 = \th(\lambda'_0)$, $\th_1,\ldots,\th_N$ are the
latitudinal turning points between $\lambda'_0$ and $\lambda'$
and $\eps_\th^{(j)}$ is the value of $\eps_\th$ between $\th_j$ and $\th_{j+1}$
for $0\leq j \leq N$,
with the convention $\th_{N+1}:=\th$.

%\subsection{Equations of motion in terms of 3+1 Kerr coordinates}

%It is interesting to write the first-order equations of geodesic motion
%in terms of the 3+1 Kerr coordinates $(\ti,r,\th,\tph)$ introduced in Sec.~\ref{s:ker:3p1_Kerr_coord}, since these coordinates are regular on the event
%horizon $\Hor$, contrary to the Boyer-Lindquist coordinates $(t,r,\th,\ph)$
%considered above.


\begin{hist}
The constant of motion $\mathscr{K}$ has been discovered by Brandon Carter\index{Carter, B.}
in 1968 \cite{Carte68}, in a study about  Kerr-Newman spacetimes\index{Kerr-Newman!spacetime},
which generalize the Kerr ones to nonzero global electric charge.
Carter actually did not get $\mathscr{K}$ from the Killing tensor $\w{K}$,
which was discovered two years later by Martin Walker\index{Walker, M.} and
Roger Penrose\index{Penrose, R.} \cite{WalkeP70}; he
started instead from the Lagrangian (\ref{e:geo:Lagrangian2}) governing
both timelike and null geodesics,
derived the corresponding
Hamiltonian as $H = g^{\mu\nu} p_\mu p_\nu$ (for the uncharged (i.e. Kerr) case)
and discovered that the Hamilton-Jacobi equation\index{Hamilton-Jacobi equation}
is separable, i.e.
can be solved by separation of variables, provided one uses the
Kerr coordinates
$(v,r,\th,\tph)$ described in Sec.~\ref{s:ker:Kerr_coord}
(NB: Carter's $u$ is our $v$) and not the Boyer-Lindquist ones.
$\mathscr{K}$ appeared then as a separation constant. Carter also introduced
the constant $Q$ via Eq.~(\ref{e:gek:def_Q}). Carter obtained the equivalent
of the first-order system (\ref{e:gek:eom_first_order}) for the Kerr coordinates. Actually
two of his equations, those for $\D r/\D\lambda$ and $\D\th/\D\lambda$, are
identical to Eqs.~(\ref{e:gek:drdl}) and (\ref{e:gek:dthdl}) of the Boyer-Lindquist
system. This is not surprising since the coordinates
$r$ and $\th$ are the same in both systems. Carter's equations for the
Kerr coordinates $v$ and $\tph$ are slightly more complicated than
Eqs.~(\ref{e:gek:dtdl}) and (\ref{e:gek:dphdl}) for the Boyer-Lindquist coordinates
$t$ and $\ph$. It seems that the Boyer-Lindquist first-order system (\ref{e:gek:eom_first_order})
has been first derived by Daniel Wilkins\index{Wilkins, D.} in 1972 \cite{Wilki72},
starting from Carter's system and performing the transformation to Boyer-Lindquist
coordinates.
\end{hist}

\section{General properties of geodesics}

\subsection{Sign of $E$}

We have:
\begin{greybox}
If the geodesic $\Li$ has some part lying outside the ergoregion $\mathscr{G}$ (cf. Sec.~\ref{s:ker:ergoregion}),
then the conserved energy $E$ defined by Eq.~(\ref{e:gek:def_E}) is necessarily positive:
\be \label{e:gek:E_positive}
    \Li \not\subset \mathscr{G}\quad \Longrightarrow \quad E > 0 .
\ee
\end{greybox}
\begin{proof}
By the very definition of the ergoregion $\mathscr{G}$ (cf. Sec.~\ref{s:ker:ergoregion}),
the Killing vector $\w{\xi}$ is timelike outside $\mathscr{G}$. Moreover it is
future-directed there, given the time orientation defined in Sec.~\ref{s:ker:Kerr_coord}.
The 4-momentum $\w{p}$ is either timelike or null and always future-directed.
By Eq.~(\ref{e:fra:scalar_caus1}), one has then necessarily $\w{\xi}\cdot\w{p} < 0$; hence
$E := - \w{\xi}\cdot\w{p} > 0$ in $\M\setminus \mathscr{G}$. Since $E$ is constant along $\Li$, it
follows that $E > 0$ everywhere.
\end{proof}
In particular, any timelike or null geodesic that reaches one of the asymptotic regions
$r\to\pm\infty$ has $E>0$.
\begin{remark}
The property (\ref{e:gek:E_positive}) generalizes that obtained for Schwarzschild spacetime [cf. Eq.~(\ref{e:ges:E_positive_M_I})] to
the case $a\not=0$, since for Schwarzschild spacetime, the exterior of the ergoregion
is nothing but the exterior of the black hole region.
\end{remark}

Inside the ergoregion, the Killing vector $\w{\xi}$ is spacelike and $E$ can be
either positive, zero or negative. Particles for which $E<0$ are called
\defin{negative-energy particles}\index{negative-energy particle}\index{particle!negative-energy --}. They are those involved in the Penrose process discussed in Sec.~\ref{s:ker:Penrose_proc}.
Note that according to (\ref{e:gek:E_positive}) a negative-energy particle cannot exist
outside the ergoregion.

\subsection{Winding near the event horizon and the inner horizon}

Let us consider a null or timelike geodesic $\Li$ in the vicinity of the
black hole event horizon $\Hor$.
On $\Hor$, $r=r_+$ and $\Delta = 0$. Then, the term
$\Delta^{-1}$ in Eq.~(\ref{e:gek:dtdl}) makes $\D t/\D\lambda$ diverge as
$r\to r_+$, except in the very special where $(r_+^2 + a^2)E - aL = 0$, which
is equivalent to $E = \Omega_H L$ according to Eq.~(\ref{e:ker:def_OmegaH}).
Similarly, $\D\ph/\D\lambda$, as given by Eq.~(\ref{e:gek:dphdl}), diverges
as $r\to r_+$, except for $E = \Omega_H L$. These two divergences are not
a pathology of $\Li$ per se; they
reflect merely the singularity of Boyer-Lindquist coordinates $(t,r,\th,\ph)$ at $\Hor$
(cf.  Sec.~\ref{s:ker:singularities}). However, we read on Eqs.~(\ref{e:gek:dtdl}) and
(\ref{e:gek:dphdl}) that the ratio
$\left. \D\ph/\D t \right| _{\Li} := \D\ph/\D\lambda \times (\D t/\D\lambda)^{-1}$
converges to a finite value:
\be \label{e:gek:lim_dphdt_Hor}
    \lim_{r\to r_+} \left. \derd{\ph}{t} \right| _{\Li} = \frac{a}{r_+^2 + a^2} = \Omega_H ,
\ee
where the second equality follows from Eq.~(\ref{e:ker:def_OmegaH}), letting
the black hole rotation velocity $\Omega_H$ appear.
Hence we conclude that
\begin{greybox}
Any null or timelike geodesic that approaches the event horizon $\Hor$
is winding around it in terms of the Boyer-Lindquist coordinates at exactly the black hole rotation velocity $\Omega_H$.
\end{greybox}
If $\Li$ is a timelike geodesic, we have argued in Sec.~\ref{s:ges:circular_orbits}
that $\left. \D\ph/\D t \right| _{\Li}$ can be interpreted as the angular velocity of
$\Li$ as seen by an asymptotic ($r\to +\infty$) inertial observer. The reasoning was
given in the Schwarzschild spacetime context but it actually  involved only
the spacetime symmetry by translation in $t$, so it is applicable here.

\begin{remark}
When $a\not=0$,
a geodesic that starts far from the black hole with
$\left. \D\ph/\D t \right| _{\Li} < 0$ [according to Eq.~(\ref{e:gek:dphdl}) with $r\gg m$,
this occurs for $L <0$]
must necessarily have a turning point in $\ph$
if it reaches the event horizon, in order to fulfill (\ref{e:gek:lim_dphdt_Hor}),
where $\Omega_H$ is positive.
This is in sharp contrast with the Schwarzschild case, where $\ph$ was always
a monotonous function of $\lambda$, as shown in Sec.~\ref{s:ges:eq_to_be_solved}.
\end{remark}

\begin{remark}
The winding property does not hold for the Kerr or 3+1 Kerr coordinates. Indeed,
we have seen in Sec.~\ref{s:ker:principal_geod} that the
ingoing principal null geodesics $\Li^{\rm in}_{(v,\th,\tph)}$ are geodesics
along which $\tph$ is constant. They are therefore not winding in terms
of neither the Kerr coordinates $(v,r,\th,\tph)$ nor the 3+1 Kerr ones $(\ti,r,\th,\tph)$.
This difference of (coordinate) behaviour is understandable if one considers
the diverging behaviour in $\Delta^{-1}$ of the
relation $\D \tph = \D \ph + a/\Delta\, \D r$ [Eq.~(\ref{e:ker:Kerr_coord_tph})]
between the angular coordinates $\ph$ and $\tph$.
\end{remark}

Similarly, considering the second root $r_-$ of $\Delta$, we deduce
from
Eqs.~(\ref{e:gek:dtdl}) and
(\ref{e:gek:dphdl}) that
\be
    \lim_{r\to r_-} \left. \derd{\ph}{t} \right| _{\Li} = \frac{a}{r_-^2 + a^2} = \Omega_{\rm in} ,
\ee
where the second equality follows from Eq.~(\ref{e:ker:def_Omega_in}); it involves
the rotation velocity $\Omega_{\rm in}$ of the inner horizon $\Hor_{\rm in}$.
Hence
\begin{greybox}
Any null or timelike geodesic that approaches the inner horizon $\Hor_{\rm in}$
is winding around it in terms of the Boyer-Lindquist coordinates at exactly the
rotation velocity $\Omega_{\rm in}$ of $\Hor_{\rm in}$.
\end{greybox}

\subsection{$\th$-motion}

The property $\Theta(\th) \geq 0$ [Eq.~(\ref{e:gek:R_Theta_positive})]
constrains the geodesic motion in $\th$. Given the expression (\ref{e:gek:Theta_Q})
for $\Theta(\th)$, this constraint can be written as
\be \label{e:gek:cond_th_motion}
    \cos^2\th \left[ a^2 (E^2 - \mu^2)
    - \frac{L^2}{\sin^2\th} \right] \geq - Q.
\ee
One may then consider three cases, depending on the sign of the Carter constant $Q$.

\subsubsection{Case $Q > 0$}

For $\th=\pi/2$ (the equator), Eq.~(\ref{e:gek:cond_th_motion}) reduces to $0 \geq -Q$.
Therefore, if $Q > 0$, there always exists an interval $[\pi/2-\alpha, \pi/2+\alpha]$
($0<\alpha \leq\pi/2$)
around $\pi/2$ such that (\ref{e:gek:cond_th_motion}) is fulfilled for
$\th$ in that interval, whatever the values
of $E$, $\mu$ and $L$. If $L\not=0$, this interval cannot be extended to
full range of $\theta$, namely $[0,\pi]$, since the left-hand side of (\ref{e:gek:cond_th_motion})
is dominated by $-L^2/\sin^2\th$ when $\th\to 0$ or $\pi$, and this term diverges to $-\infty$.
Even when $L=0$, (\ref{e:gek:cond_th_motion}) allows for $\th=0$ or $\pi$
only if $a^2 (E^2 - \mu^2) \geq - Q$.
This condition is always satisfied for a null geodesic ($\mu=0$) but may
fail for a timelike one if $E^2 < \mu^2$.
We conclude that
\begin{greybox}
A geodesic with a positive Carter constant $Q$ has a $\th$-motion of limited
amplitude $\th\in [\pi/2-\alpha, \pi/2+\alpha]$ ($0<\alpha<\pi/2$)
around the equatorial plane, except if
\be \label{e:gek:cond_reach_axis}
    L = 0 \quad\mbox{and}\quad a^2 (E^2 - \mu^2) \geq - Q ,
\ee
in which case one has $\alpha=\pi/2$, so that the geodesic may reach the rotation axis.
\end{greybox}
\begin{remark}
For $L=0$ and values of $E$, $\mu$ and $Q$ saturating the inequality in (\ref{e:gek:cond_reach_axis}),
i.e. obeying $0 < Q = a^2 (\mu^2 - E^2)$, one has $\Theta(\th) = Q(1-\cos^2\th)$,
so that $\Theta(0) = 0$ and $\Theta(\pi) = 0$. A geodesic along the rotation
axis is then a solution.
\end{remark}

\subsubsection{Case $Q = 0$}

\subsubsection{Case $Q < 0$}

\subsection{$r$-motion}

\subsection{Geodesics reaching or emanating from the ring singularity}

\subsection{Moving to the negative $r$ side}


\section{Timelike geodesics}

\subsection{Parametrization}

\subsection{Bound geodesics}

\subsection{Equatorial geodesics}

%%%%%%%%%%%%%%%%%%%%%%%%%%%%%%%%%%%%%%%%%%%%%%%%%%%%%%%%%%%%%%%%%%%%%%%%%%%%%%%

\section{Null geodesics}

\subsection{Parametrization}

\subsection{Principal null geodesics}

\subsection{Spherical null geodesics}






Circular orbits in Kerr spacetime: \cite{BardePT72} (see also \cite{Barde73})

More general orbits: \cite{Perez-Giz11}, \cite{GrossLP12}
