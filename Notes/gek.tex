\chapter{Geodesics in Kerr spacetime}
\label{s:gek}
\index{geodesic!in Kerr spacetime}

\minitoc

\section{Introduction}

\section{Equations of geodesic motion}

In all this chapter, we are concerned by the motion of a particle
$\mathscr{P}$ in Kerr spacetime $(\M,\w{g})$, assuming that $\mathscr{P}$
feels only gravity, as described by $\w{g}$ (freely falling particle).
The worldline $\Li$ of $\mathscr{P}$
is then necessarily a geodesic\footnote{The definition and basic properties of geodesics
are recalled in Appendix~\ref{s:geo}; see also Sec.~\ref{s:fra:geod_motion}.} of
$(\M,\w{g})$. It is a timelike geodesic if $\mathscr{P}$ is a massive particle
and a null geodesic if $\mathscr{P}$ is massless (e.g. a photon).

The starting point for the study of geodesics in Kerr spacetime is pretty similar
to that of geodesics in Schwarzschild spacetime performed in Chap.~\ref{s:ges},
since both spacetimes are stationary (outside the ergosphere) and axisymmetric,
which gives birth to two first integrals of motion, $E$ and $L$ (Sec.~\ref{s:gek:int_motion_sym}).
A major difference is that in the Kerr case, generic geodesics are not
planar, i.e. are not confined
to a hypersurface, contrary to what happens in Schwarzschild spacetime, where
thanks to the spherical symmetry of the latter, a suitable choice of
coordinates $(t,r,\th,\ph)$ makes a given geodesic confined to the
hyperplane $\th=\pi/2$. We therefore lose the first integral $p^\th=0$
[Eq.~(\ref{e:ges:pth_zero})]. Fortunaly there exists another
first integral of motion, the Carter constant, due to a remarkable property
of Kerr spacetime: the existence of a non-trivial Killing tensor (Sec.~\ref{s:gek:Carter_const}).
With the particle mass $\mu$, this makes a total of four integral of motions,
which renders the problem integrable, reducing it to a set of four
first-order differential equations (Sec.~\ref{s:gek:first_order_system}).


\subsection{Integrals of motion from symmetries} \label{s:gek:int_motion_sym}

As for Schwarzschild spacetime (cf. Sec.~\ref{s:ges:fiom}), we have the following
property.
\begin{greybox}
The Killing vectors $\w{\xi}$ and $\w{\eta}$ of Kerr spacetime,
associated respectively with
stationarity and axisymmetry [cf. Eq.~(\ref{e:ker:def_xi_eta})],
give birth to two conserved quantities
along the geodesic $\Li$:
\begin{subequations}
\label{e:gek:def_E_L}
\begin{align}
& \encadre{E := - \w{\xi}\cdot \w{p} = - \w{g}(\w{\xi},\w{p}) } \label{e:gek:def_E} \\
& \encadre{L := \w{\eta}\cdot \w{p} = \w{g}(\w{\eta},\w{p}) } , \label{e:gek:def_L}
\end{align}
\end{subequations}
where $\w{p}$ is the 4-momentum of particle $\mathscr{P}$ (cf. Sec.~\ref{s:fra:worldlines}).
For the same reasons as in Sec.~\ref{s:ges:fiom}, $E$ is called
the \defin{conserved energy}\index{conserved!energy}\index{energy!conserved --}
or \defin{energy at infinity}\index{energy!at infinity} of $\mathscr{P}$,
while $L$ is called $\mathscr{P}$'s \defin{conserved angular momentum}\index{conserved!angular momentum}\index{angular momentum!conserved --}
or \defin{angular momentum at infinity}\index{angular momentum!at infinity}
of $\mathscr{P}$.
\end{greybox}
\begin{remark}
Asymptotically, the scalar $L$ is only the
component along the rotation axis of $\mathscr{P}$'s angular momentum vector $\w{L}$
as measured by an inertial observer at rest with respect to the black hole.
For this reason, it is sometimes denoted by $L_z$, instead of merely $L$.
\end{remark}

In coordinates $(t,r,\th,\ph)$ adapted to the spacetime symmetries,
i.e. coordinates such that $\w{\xi} = \wpar_t$ and $\w{\eta}=\wpar_\ph$,
for instance Boyer-Lindquist coordinates (Sec.~\ref{s:ker:expr_BL}),
Kerr coordinates (Sec.~\ref{s:ker:Kerr_coord}) or 3+1 Kerr coordinates
(Sec.~\ref{s:ker:3p1_Kerr_coord}), one can rewrite
(\ref{e:gek:def_E_L})
in terms of the components $p_t = g_{t\mu} \, p^\mu$ and $p_\ph = g_{\ph\mu} \, p^\mu$
of the 1-form $\uu{p}$ associated to $\w{p}$ by metric duality:
\begin{subequations}
\begin{align}
& E = - p_t \\
& L = p_\ph
\end{align}
\end{subequations}
Indeed, in such a coordinate system, $\xi^\mu =  \delta^\mu_{\ \, t}$
and $\eta^\mu = \delta^\mu_{\ \, \ph}$, so that $E = -g_{\mu\nu} \, \xi^\mu p^\nu = -g_{t\nu} \, p^\nu = -p_t$
and $L = g_{\mu\nu} \, \eta^\mu p^\nu = g_{\ph\nu} \, p^\nu = p_\ph$.

In what follows, we are going to use Boyer-Lindquist coordinates
$(x^\alpha)=(t,r,\th,\ph)$
as introduced in Sec.~\ref{s:ker:expr_BL}.
Given the components (\ref{e:ker:metric_BL}) of the metric tensor $\w{g}$
in these coordinates, evaluating $E$ and $L$
via $E = - g_{t\mu} \, p^\mu$ and $L = g_{\ph\mu} \, p^\mu$ yields
\be
    E = \left( 1 - \frac{2 m r}{\rho^2} \right)\,  p^t
        + \frac{2 a m r \sin^2\th}{\rho^2}\,  p^\ph  .
\ee
\be
    L = - \frac{2 a m r \sin^2\th}{\rho^2} \, p^t
        + \left( r^2 + a^2 + \frac{2 a^2 m r \sin^2\th}{\rho^2} \right)
    \sin^2\th \,  p^\ph ,
\ee
where $\rho^2 := r^2 + a^2 \cos^2\th$ [Eq.~(\ref{e:ker:def_rho2})].

Let us recall that the components $(p^\alpha)$ of the 4-momentum are
related to the equation $x^\alpha = x^\alpha(\lambda)$ of the geodesic $\Li$
in terms of the affine parameter $\lambda$ by $p^\alpha = \D x^\alpha / \D\lambda$
[cf. Eq.~(\ref{e:fra:p_dxdl})], i.e.
\be
    p^t = \derd{t}{\lambda},\quad
    p^r = \derd{r}{\lambda},\quad
    p^\th = \derd{\th}{\lambda},\quad
    p^\ph = \derd{\ph}{\lambda} .
\ee
In other words, $\lambda$ is the affine parameter\footnote{See Sec.~\ref{s:geo:def} for the definition
of an affine parameter along a geodesic.} along $\Li$ such that the tangent vector
to $\Li$ associated with $\lambda$ is $\w{p}$.

\subsection{Mass as an integral of motion}

The mass $\mu$ of particle $\mathscr{P}$ is given by the scalar square of
the 4-momentum vector $\w{p}$ via Eq.~(\ref{e:fra:def_mass}):
\be \label{e:gek:mu_gpp}
    \mu^2 = - \w{g}(\w{p}, \w{p}) .
\ee
The fact that $\Li$ is a geodesic implies that $\mu$ is a constant along
$\Li$ (cf. Eq.~(\ref{e:geo:vv_const}) in Appendix~\ref{s:geo}). It therefore
provides a third integral of motion, after $E$ and $L$.

Let us express (\ref{e:gek:mu_gpp}) in terms of the inverse metric
in order to let appear $p_t = -E$ and $p_\ph = L$:
\[
    \mu^2 = - g^{\mu\nu} p_\mu p_\nu .
\]

Given the components (\ref{e:ker:inv_met_BL}) of the inverse metric
in Boyer-Lindquist coordinates, we get
\bea
    \mu^2 & = & \frac{1}{\Delta}
    \left( r^2 + a^2 + \frac{2 a^2 m r \sin^2\th}{\rho^2} \right) E^2
    -\frac{4 a m r}{\rho^2 \Delta} E L
    - \frac{1}{\Delta}\left(1 - \frac{2 m r}{\rho^2} \right) \frac{L^2}{\sin^2\th}
    \nonumber \\
   &  & - \frac{\rho^2}{\Delta} \left( p^r \right) ^2
    - \rho^2 \left( p^\theta \right) ^2 ,
\eea
where $\Delta := r^2 - 2 m r + a^2$ [Eq.~(\ref{e:ker:def_Delta})].
Note that we have expressed $p_r$ and $p_\th$ in terms of $p^r$ and $p^\th$
thanks to the relations $p_r = g_{r\mu} p^\mu$ and $p_\th = g_{\th\mu} p^\mu$,
which are very simple for the Boyer-Lindquist components (\ref{e:ker:metric_BL})
of $\w{g}$:
\be
    p_r = \frac{\rho^2}{\Delta} \, p^r
    \qquad\mbox{and}\qquad
    p_\th = \rho^2 \, p^\th .
\ee

\subsection{The fourth integral of motion: Carter constant} \label{s:gek:Carter_const}

It turns out that the Kerr spacetime is endowed with a non-trivial
Killing tensor of valence 2: the
\defin{Walker-Penrose Killing tensor}\index{Killing!tensor!Walker-Penrose --}\index{Walker-Penrose Killing tensor} $\w{K}$, which is the symmetric tensor of type
$(0,2)$ defined by
\be
    \encadre{ \w{K} := (r^2 + a^2) \left( \uu{k} \otimes \uu{\el} = \uu{\el} \otimes \uu{k} \right) + r^2 \w{g} } ,
\ee
where $\uu{\el}$ and $\uu{k}$ are the 1-forms associated by metric duality
to the null vector fields $\w{k}$ and $\wl$ tangent to the principal null
geodesics introduced in Sec.~\ref{s:ker:principal_geod}.

\subsection{First order equations of motion} \label{s:gek:first_order_system}


\section{Motion in the equatorial plane}

Circular orbits in Kerr spacetime: \cite{BardePT72} (see also \cite{Barde73})

More general orbits: \cite{Perez-Giz11}, \cite{GrossLP12}
