\chapter{Geodesics in Kerr spacetime}
\label{s:gek}
\index{geodesic!in Kerr spacetime}

\minitoc

\section{Introduction}

\section{Equations of geodesic motion}

\subsection{Introduction}

In all this chapter, we are concerned by the motion of a particle
$\mathscr{P}$ in Kerr spacetime $(\M,\w{g})$, assuming that $\mathscr{P}$
feels only gravity, as described by $\w{g}$ (freely falling particle).
The worldline $\Li$ of $\mathscr{P}$
is then necessarily a geodesic\footnote{The definition and basic properties of geodesics
are recalled in Appendix~\ref{s:geo}; see also Sec.~\ref{s:fra:geod_motion}.} of
$(\M,\w{g})$. It is a timelike geodesic if $\mathscr{P}$ is a massive particle
and a null geodesic if $\mathscr{P}$ is massless (e.g. a photon).

In a given coordinate system $(x^\alpha)$, the geodesic $\Li$ is described
by a system of the form\footnote{In Appendices~\ref{s:bas} and \ref{s:geo},
we use a different notation for the function $X^\alpha(\lambda)$ defining
$\Li$ and the coordinate $x^\alpha$ (cf. Secs.~\ref{s:bas:vectors} and
\ref{s:geo:geodesic_eq}). Following standard usage in the physics literature,
we shall not do this here.}
$x^\alpha = x^\alpha(\lambda)$, where $\lambda$ is an
affine parameter\footnote{See Sec.~\ref{s:geo:def} for the definition
of an affine parameter along a geodesic.} along $\Li$. We choose $\lambda$ to be the affine parameter
associated with $\mathscr{P}$'s 4-momentum $\w{p}$ [cf. Eq.~(\ref{e:fra:p_dxdl})]. We have
then $\D x^\alpha / \D\lambda = p^\alpha$.
The curve $\Li$ is a geodesic iff the functions $x^\alpha(\lambda)$ obey
the geodesic equation [Eq.~(\ref{e:geo:eq_geod}) in Appendix~\ref{s:geo}]:
\be \label{e:gek:eq_geod}
    \frac{\D^2 x^\alpha}{\D\lambda^2} + \Gamma^\alpha_{\ \, \mu \nu}
    \frac{\D x^\mu}{\D\lambda} \frac{\D x^\nu}{\D\lambda} = 0 ,
\ee
where the $\Gamma^\alpha_{\ \, \mu \nu}$'s are the Christoffel symbols\index{Christoffel symbols} of the metric $\w{g}$
with respect to the coordinates $(x^\alpha)$, as given by Eq.~(\ref{e:bas:Christoffel}).
Equation~(\ref{e:gek:eq_geod}) is actually a system of four coupled second-order
differential equations, that are non-linear. We are going to see that in
order to compute the geodesics of Kerr spacetime, it is not necessary to
solve this system, since, as in the Schwarzschild case studied in Chap.~\ref{s:ges}, we have
enough first integrals of Eq.~(\ref{e:gek:eq_geod}) to reduce
the problem to four first-order equations
of the type $\D x^\alpha/\D\lambda = F^\alpha(x^0, x^1, x^2, x^3)$.

Two integrals of motion are similar to those in the Schwarzschild case,
since both Kerr and Schwarzschild spacetimes are stationary (outside the ergosphere) and axisymmetric,
which gives birth to the conserved energy $E$ and conserver angular momentum
$L$ (Sec.~\ref{s:gek:int_motion_sym}).
A major difference is that in the Kerr case, generic geodesics are not
planar, i.e. are not confined
to a hypersurface, contrary to what happens in Schwarzschild spacetime, where
thanks to the spherical symmetry of the latter, a suitable choice of
coordinates $(t,r,\th,\ph)$ makes a given geodesic confined to the
hyperplane $\th=\pi/2$. We therefore lose the first integral $p^\th=0$
[Eq.~(\ref{e:ges:pth_zero})]. Fortunaly there exists another
integral of motion, the Carter constant, due to a remarkable property
of Kerr spacetime: the existence of a non-trivial Killing tensor (Sec.~\ref{s:gek:Carter_const}).
With the particle mass $\mu$, this makes a total of four integral of motions,
which renders the problem integrable (Sec.~\ref{s:gek:first_order_system}).


\subsection{Integrals of motion from symmetries} \label{s:gek:int_motion_sym}

As for Schwarzschild spacetime (cf. Sec.~\ref{s:ges:fiom}), we have the following
property.
\begin{greybox}
The Killing vectors $\w{\xi}$ and $\w{\eta}$ of Kerr spacetime,
associated respectively with
stationarity and axisymmetry [cf. Eq.~(\ref{e:ker:def_xi_eta})],
give birth to two conserved quantities
along the geodesic $\Li$:
\begin{subequations}
\label{e:gek:def_E_L}
\begin{align}
& \encadre{E := - \w{\xi}\cdot \w{p} = - \w{g}(\w{\xi},\w{p}) } \label{e:gek:def_E} \\
& \encadre{L := \w{\eta}\cdot \w{p} = \w{g}(\w{\eta},\w{p}) } , \label{e:gek:def_L}
\end{align}
\end{subequations}
where $\w{p}$ is the 4-momentum of particle $\mathscr{P}$ (cf. Sec.~\ref{s:fra:worldlines}).
For the same reasons as in Sec.~\ref{s:ges:fiom}, $E$ is called
the \defin{conserved energy}\index{conserved!energy}\index{energy!conserved --}
or \defin{energy at infinity}\index{energy!at infinity} of $\mathscr{P}$,
while $L$ is called $\mathscr{P}$'s \defin{conserved angular momentum}\index{conserved!angular momentum}\index{angular momentum!conserved --}
or \defin{angular momentum at infinity}\index{angular momentum!at infinity}
of $\mathscr{P}$.
\end{greybox}
\begin{remark}
Asymptotically, the scalar $L$ is only the
component along the rotation axis of $\mathscr{P}$'s angular momentum vector $\w{L}$
as measured by an inertial observer at rest with respect to the black hole.
For this reason, it is sometimes denoted by $L_z$, instead of merely $L$.
\end{remark}

In coordinates $(t,r,\th,\ph)$ adapted to the spacetime symmetries,
i.e. coordinates such that $\w{\xi} = \wpar_t$ and $\w{\eta}=\wpar_\ph$,
for instance Boyer-Lindquist coordinates (Sec.~\ref{s:ker:expr_BL}),
Kerr coordinates (Sec.~\ref{s:ker:Kerr_coord}) or 3+1 Kerr coordinates
(Sec.~\ref{s:ker:3p1_Kerr_coord}), one can rewrite
(\ref{e:gek:def_E_L})
in terms of the components $p_t = g_{t\mu} \, p^\mu$ and $p_\ph = g_{\ph\mu} \, p^\mu$
of the 1-form $\uu{p}$ associated to $\w{p}$ by metric duality:
\begin{subequations}
\label{e:gek:E_pt_L_pph}
\begin{align}
& E = - p_t \\
& L = p_\ph
\end{align}
\end{subequations}
Indeed, in such a coordinate system, $\xi^\mu =  \delta^\mu_{\ \, t}$
and $\eta^\mu = \delta^\mu_{\ \, \ph}$, so that $E = -g_{\mu\nu} \, \xi^\mu p^\nu = -g_{t\nu} \, p^\nu = -p_t$
and $L = g_{\mu\nu} \, \eta^\mu p^\nu = g_{\ph\nu} \, p^\nu = p_\ph$.

In what follows, we are going to use Boyer-Lindquist coordinates
$(x^\alpha)=(t,r,\th,\ph)$
as introduced in Sec.~\ref{s:ker:expr_BL}.
Given the components (\ref{e:ker:metric_BL}) of the metric tensor $\w{g}$
in these coordinates, evaluating $E$ and $L$
via $E = - g_{t\mu} \, p^\mu$ and $L = g_{\ph\mu} \, p^\mu$ yields
\be \label{e:gek:E_first_int}
    E = \left( 1 - \frac{2 m r}{\rho^2} \right)\,  p^t
        + \frac{2 a m r \sin^2\th}{\rho^2}\,  p^\ph  .
\ee
\be \label{e:gek:L_first_int}
    L = - \frac{2 a m r \sin^2\th}{\rho^2} \, p^t
        + \left( r^2 + a^2 + \frac{2 a^2 m r \sin^2\th}{\rho^2} \right)
    \sin^2\th \,  p^\ph ,
\ee
where $\rho^2 := r^2 + a^2 \cos^2\th$ [Eq.~(\ref{e:ker:def_rho2})].

Let us recall that the components $(p^\alpha)$ of the 4-momentum are
related to the equation $x^\alpha = x^\alpha(\lambda)$ of the geodesic $\Li$
in terms of the affine parameter $\lambda$ by $p^\alpha = \D x^\alpha / \D\lambda$
[cf. Eq.~(\ref{e:fra:p_dxdl})], i.e.
\be \label{e:gek:pa_der_xa}
    p^t = \derd{t}{\lambda},\quad
    p^r = \derd{r}{\lambda},\quad
    p^\th = \derd{\th}{\lambda},\quad
    p^\ph = \derd{\ph}{\lambda} .
\ee

\subsection{Mass as an integral of motion}

The mass $\mu$ of particle $\mathscr{P}$ is related to the scalar square of
the 4-momentum vector $\w{p}$ via Eq.~(\ref{e:fra:def_mass}):
\be \label{e:gek:mu_gpp}
    \mu^2 = - \w{g}(\w{p}, \w{p}) .
\ee
The fact that $\Li$ is a geodesic implies that $\mu$ is a constant along
$\Li$ (cf. Eq.~(\ref{e:geo:vv_const}) in Appendix~\ref{s:geo}). It therefore
provides a third integral of motion, after $E$ and $L$.

Let us express (\ref{e:gek:mu_gpp}) in terms of the inverse metric
in order to let appear $p_t = -E$ and $p_\ph = L$:
\[
    \mu^2 = - g^{\mu\nu} p_\mu p_\nu .
\]

Given the components (\ref{e:ker:inv_met_BL}) of the inverse metric
in Boyer-Lindquist coordinates, we get
\bea
    \mu^2 & = & \frac{1}{\Delta}
    \left( r^2 + a^2 + \frac{2 a^2 m r \sin^2\th}{\rho^2} \right) E^2
    -\frac{4 a m r}{\rho^2 \Delta} E L
    - \frac{1}{\Delta}\left(1 - \frac{2 m r}{\rho^2} \right) \frac{L^2}{\sin^2\th}
    \nonumber \\
   &  & - \frac{\rho^2}{\Delta} \left( p^r \right) ^2
    - \rho^2 \left( p^\theta \right) ^2 ,   \label{e:gek:mu2_first_int}
\eea
where $\Delta := r^2 - 2 m r + a^2$ [Eq.~(\ref{e:ker:def_Delta})].
Note that we have expressed $p_r$ and $p_\th$ in terms of $p^r$ and $p^\th$
thanks to the relations $p_r = g_{r\mu} p^\mu$ and $p_\th = g_{\th\mu} p^\mu$,
which are very simple for the Boyer-Lindquist components (\ref{e:ker:metric_BL})
of $\w{g}$:
\be \label{e:gek:p_r_p_th_cov_con}
    p_r = \frac{\rho^2}{\Delta} \, p^r
    \qquad\mbox{and}\qquad
    p_\th = \rho^2 \, p^\th .
\ee

\subsection{The fourth integral of motion: Carter constant} \label{s:gek:Carter_const}

It turns out that the Kerr spacetime is endowed with a non-trivial
Killing tensor of valence 2: the
\defin{Walker-Penrose Killing tensor}\index{Killing!tensor!Walker-Penrose --}\index{Walker-Penrose Killing tensor} $\w{K}$, which is the symmetric tensor of type
$(0,2)$ defined by
\be \label{e:gek:def_K}
    \encadre{ \w{K} := (r^2 + a^2) \left( \uu{k} \otimes \uu{\el} + \uu{\el} \otimes \uu{k} \right) + r^2 \w{g} } ,
\ee
where $\uu{\el}$ and $\uu{k}$ are the 1-forms associated by metric duality
to the null vector fields $\w{k}$ and $\wl$ tangent to the principal null
geodesics introduced in Sec.~\ref{s:ker:principal_geod}. In index notation,
Eq.~(\ref{e:gek:def_K}) writes
\be
    K_{\alpha\beta} = (r^2 + a^2) \left(k_\alpha \el_\beta + \el_\alpha k_\beta \right)
        + r^2 g_{\alpha\beta} .
\ee
$\w{K}$ is called a \defin{Killing tensor}\index{Killing!tensor} because its symmetrized covariant derivative vanishes identically:
\be \label{e:gek:Killing_eq_K}
    \encadre{ \nabla_{(\alpha} K_{\beta\gamma)} = 0 }.
\ee
This property can be seen as a generalization of the Killing equation
(\ref{e:neh:Killing_equation}) to tensors of valence 2.
That the tensor $\w{K}$ defined by (\ref{e:gek:def_K}) obeys the Killing
identity (\ref{e:gek:Killing_eq_K}) is shown in the notebook~\ref{s:sam:Kerr_Killing_tensor}.

Killing tensors are discussed in Sec.~\ref{e:geo:Killing_tensor} of Appendix~\ref{s:geo}.
It is shown there that the Killing identity (\ref{e:gek:Killing_eq_K}) implies that
the following quantity is constant along any geodesic $\Li$ [cf. Eq.~(\ref{e:geo:Kvv_const})]:
\be
    \encadre{ \mathscr{K} := \w{K}(\w{p}, \w{p}) = K_{\mu\nu} p^\mu p^\nu } .
\ee
$\mathscr{K}$ is named \defin{Carter constant}\index{Carter!constant}.
From the definition (\ref{e:gek:def_K}) of $\w{K}$, we have
\be \label{e:gek:Kcarter_prov}
    \mathscr{K} = 2(r^2 + a^2) \langle \uu{k}, \w{p} \rangle
        \langle \uu{\el}, \w{p} \rangle + r^2 \w{g}(\w{p}, \w{p}).
\ee
Now by Eq.~(\ref{e:gek:mu_gpp}), $\w{g}(\w{p}, \w{p}) = - \mu^2$. Besides, we
have
\[
  \langle \uu{k}, \w{p} \rangle = k_\mu p^\mu = k^\mu p_\mu .
\]
The last form allows us to let appear the constants of motion $p_t = -E$ and
$p_\ph = L$ [Eq.~(\ref{e:gek:E_pt_L_pph})]. Using it with the components
of $\w{k}$ as given by Eq.~(\ref{e:ker:k_BL}), we get
\[
    \langle \uu{k}, \w{p} \rangle = - \frac{r^2 + a^2}{\Delta} E
    - p_r + \frac{a}{\Delta} L .
\]
Similarly, from the components (\ref{e:ker:ell_BL}) of $\wl$, we obtain
\[
     \langle \uu{\el}, \w{p} \rangle = - \frac{1}{2} E + \frac{\Delta}{2(r^2 + a^2)} p_r
     + \frac{a}{2(r^2 + a^2)} L .
\]
Accordingly, Eq.~(\ref{e:gek:Kcarter_prov}) becomes
\[
    \mathscr{K} = \frac{1}{\Delta} \left[ (r^2 + a^2) E + \Delta p_r - a L \right]
      \left[ (r^2 + a^2) E - \Delta p_r - a L \right] - r^2 \mu^2 ,
\]
which can be rewritten as
\be \label{e:gek:Kcarter_first_int}
    \mathscr{K} = \frac{1}{\Delta} \left[ \left( (r^2 + a^2) E - a L \right)^2
        - \rho^4 (p^r)^2 \right] - r^2 \mu^2 .
\ee
Note that we have expressed $p_r$ in terms of $p^r$ via Eq.~(\ref{e:gek:p_r_p_th_cov_con}).

\subsection{First order equations of motion} \label{s:gek:first_order_system}

We have thus obtained four first integrals of the geodesic equation
(\ref{e:gek:eq_geod}):
$E$ [Eq.~(\ref{e:gek:E_first_int})], $L$ [Eq.~(\ref{e:gek:L_first_int})],
$\mu^2$ [Eq.~(\ref{e:gek:mu2_first_int})] and $\mathscr{K}$
[Eq.~(\ref{e:gek:Kcarter_first_int})]. In the expressions of each of these integrals,
$p^\alpha$ has to be thought of as the first order derivative $\D x^\alpha/\D\lambda$
[Eq.~(\ref{e:gek:pa_der_xa})].
Two first integrals, namely $E$ and $L$, are linear in the $p^\alpha$'s, while
the two others, namely $\mu^2$ and $\mathscr{K}$, are quadratic.
Furthermore, Eqs.~(\ref{e:gek:E_first_int}) and (\ref{e:gek:L_first_int})
constitute a decoupled subsystem for $(p^t, p^\ph)$, which can be easily solved,
yielding
\be
    \rho^2 p^t = \frac{r^2 + a^2}{\Delta} [(r^2 + a^2) E - a L] + a ( L - a E \sin^2\th )
\ee
\be
    \rho^2 p^\ph = \frac{L}{\sin^2\th}
    + a \left[\frac{1}{\Delta}\left((r^2 + a^2) E - a L\right) - E \right] .
\ee
Besides, Eq.~(\ref{e:gek:Kcarter_first_int}) involves only $p^r$ and can
be recast as
\be
    \rho^2 p^r = \pm \sqrt{ R(r) } ,
\ee
with
\be
    R(r) := \left[ (r^2 + a^2) E - a L \right]^ 2 - \Delta (r^2 \mu^2 + \mathscr{K}) .
\ee
In the above expression, let us recall that $\Delta$ is the function of $r$
given by Eq.~(\ref{e:ker:def_Delta}): $\Delta := r^2 - 2 m r + a^2$.

\section{Motion in the equatorial plane}








Circular orbits in Kerr spacetime: \cite{BardePT72} (see also \cite{Barde73})

More general orbits: \cite{Perez-Giz11}, \cite{GrossLP12}
