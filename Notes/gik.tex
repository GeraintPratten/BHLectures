\chapter{Null geodesics and images in Kerr spacetime}
\label{s:gik}

\minitoc

\section{Introduction}

Having investigated the properties of generic causal geodesics
in Schwarzschild spacetime in Chap.~\ref{s:ges}, we focus here on null
geodesics.


\section{Main properties of null geodesics} \label{s:gik:properties}

We shall distinguish the null geodesics with $E=0$ from those having
$E \neq 0$. Indeed, in the latter case, we will rescale the angular momentum
$L$ and the Carter constant $Q$ by $E$, so that only two constants of motion become
pertinent for the study: $L/E$ and $Q/E^2$.
We thus treat first the particular case $E=0$.

\subsection{Null geodesics with $E=0$}


First, we note that a geodesic with $E=0$ cannot exist outside the ergoregion
$\mathscr{G}$, by virtue of the result (\ref{e:gek:E_positive}). In particular,
it cannot exist far from the black hole.

Another property of null geodesics with $E=0$ is a non-negative Carter constant:
\be
    Q \geq 0 .
\ee
This follows immediately from the result of Sec.~\ref{s:gek:th_motion}
stating that a necessary condition for $Q < 0$ is $a\neq 0$ and
$|E| > \sqrt{\mu^2 + L^2/a^2}$. Specilizing the latter condition to $\mu=0$
and $E=0$, we get $0 > |L|$, which is impossible.

\subsubsection{$\theta$-motion}

Specializing the general results of Sec.~\ref{s:gek:th_motion} to $\mu=0$ and $E=0$,
and taking into account $Q \geq 0$, we get
\begin{greybox}
\begin{itemize}
\item If $Q>0$, (i) for $L\neq 0$, $\Li$ oscillates symmetrically about the equatorial plane,
between two turning points, $\th_0$ and $\pi-\th_0$, such that $0<\th_0<\pi/2$;
(ii) for $L=0$, $\Li$
crosses repeatedly the rotation axis, with $\th$ taking all values in the
range $[0,\pi]$.
\item If $Q=0$, $\Li$ is stably confined to the equatorial plane
for $L \neq 0$;
for $L=0$, $\Li$ lies at a constant value $\th=\th_0\in[0,\pi]$.
\end{itemize}
\end{greybox}


\subsection{Equations of geodesic motion for $E\neq 0$}


\subsection{$\theta$-motion}

Specializing the general results of Sec.~\ref{s:gek:th_motion} to $\mu=0$, we
get
\begin{greybox}
\begin{itemize}
\item A null geodesic $\Li$ of Kerr spacetime cannot encounter the rotation axis unless it has $L=0$.
\item If $a=0$ (Schwarzschild limit) or $|E| \leq |L|/a$,
the Carter constant $Q$ is necessarily non-negative:
\be \label{e:gek:Q_nonnegative}
    Q \geq 0 .
\ee
\item The Carter constant $Q$ can take negative values only if
$a\neq 0$ and $|E| > |L|/a$; its range is then
limited from below:
\be
    Q \geq Q_{\rm min} = - \left( a |E| - |L| \right) ^2.
\ee
If $Q<0$, $\Li$ is called a \defin{vortical null geodesic}\index{vortical geodesic}; it
never encounters the equatorial plane.
\item If $Q>0$ and $L\not=0$, $\Li$ oscillates symmetrically about the equatorial plane,
between two turning points, $\th_0$ and $\pi-\th_0$, such that $0<\th_0<\pi/2$.
If $Q>0$ and $L=0$, $\Li$
crosses repeatedly the rotation axis, with $\th$ taking all values in the
range $[0,\pi]$.
\item If $Q=0$, $\Li$ is stably confined to the equatorial plane
for $a |E|  < |L|$ or $a |E|  = |L| \neq 0$;
for $a |E| > |L|$, $\Li$ either lies unstably in the equatorial
plane or approaches it asymptotically from one side, while for $L=0$ and ($a=0$ or $|E|=\mu$),
$\Li$ lies at a constant value $\th=\th_0\in[0,\pi]$.
\item If $Q_{\rm min} < Q < 0$, $\Li$ oscillates between two turning
points strictly located in the Northern hemisphere ($0<\th<\pi/2$) or in
the Southern hemisphere ($\pi/2<\th<\pi$) for $L\neq 0$, while for $L=0$,
$\Li$ oscillates about the rotation axis, without encoutering the equatorial
plane, having a turning point at $\th_0$ such that $0<\th_0<\pi/2$
(motion in the Northern hemisphere) or $\pi/2<\th_0<\pi$
(motion in the Southern hemisphere).
\item If $Q = Q_{\rm min}$, $\Li$ lies stably at a constant value $\th=\th_0$,
with $\th_0 = 0$ or $\pi$ (the rotation axis) for $L=0$ and $0<\th_0< \pi/2$
or $\pi/2<\th_0<\pi$ for $L\neq 0$.
\end{itemize}
\end{greybox}


\section{Principal null geodesics}

\section{Photon region}

\subsection{Spherical null geodesics}

\section{Images}
