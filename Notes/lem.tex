\chapter{Black hole formation 1: dust collapse}
\label{s:lem}

\minitoc

\section{Introduction}

After having investigated black holes in equilibrium, in the form of
Schwarzschild and Kerr solutions, we turn to dynamical black holes,
more specifically to the standard process of
black hole formation: \emph{gravitational collapse}\index{gravitational!collapse}.
To deal with analytical solutions, we simplify the problem as much as
possible. First we assume spherical symmetry, which is quite natural
as a first approximation for modelling the gravitational collapse
of a stellar core or a gas cloud. A drawback is that this forbids the
study of gravitational waves\index{gravitational!waves}, since by
virtue of Birkhoff's theorem\index{Birkhoff's theorem} (to be proven in this chapter)
the exterior
of any spherically symmetric collapsing object is a piece of Schwarzschild
spacetime, i.e. it does not contain any gravitational radiation.
The second major approximation is to consider \emph{pressureless matter},
commonly referred to as \emph{dust}\index{dust}.
An alternative, certainly more academic, is to consider the collapse of shell of pure electromagnetic radiation; this will be performed in Chap.~\ref{s:vai}.

\section{Lemaître-Tolman equations}

\subsection{Hypotheses} \label{s:lem:hyp}

As mentioned in the Introduction, we shall restrict ourselves to
spherically symmetric\footnote{See Sec.~\ref{s:sch:static_spher} for a precise
definition of \emph{spherically symmetric}.} spacetimes, and for concreteness, to 4-dimensional ones. The most general spherically symmetric 4-dimensional spacetime $(\M,\w{g})$
can be described in terms of coordinates $(x^\alpha)=(\tau,\chi,\th,\ph)$ such that
the metric tensor writes
\be \label{e:lem:metric_sync_coord}
    \w{g} = - \dd\tau^2 + a(\tau,\chi)^2 \dd\chi^2
        + r(\tau,\chi)^2 \left( \dd\th^2 + \sin^2\th\, \dd\ph^2 \right)  ,
\ee
where $a(\tau,\chi)$ and $r(\tau,\chi)$ are generic positive functions.
These coordinates are called \defin{Lemaître synchronous coordinates}\index{Lemaitre@Lemaître!synchronous coordinates}\index{synchronous!coordinates}, the qualifier
\defin{synchronous} meaning that $\tau$ is the proper time of a observer staying
at fixed value of the spatial coordinates $(\chi,\th,\ph)$.
Note that the function $r(\tau,\chi)$ gives the \emph{areal radius}\index{areal!radius}
of the 2-spheres
defined by $(\tau,\chi) = \mathrm{const}$, which are the orbits of the $\mathrm{SO}(3)$
group action (cf. Sec.~\ref{s:sch:static_spher}), i.e. the metric area of these
2-spheres is $4\pi r(\tau,\chi)^2$.

For simplification, we consider only a pressureless matter, in the form of a perfect
fluid of 4-velocity $\w{u}$ with zero pressure. The matter energy-momentum tensor is then
\be \label{e:lem:T_pressureless}
    \w{T} = \rho \w{u} \otimes \w{u} ,
\ee
where the scalar field $\rho$ can be interpreted as the fluid energy density
measured in the fluid frame. Let us recall that the energy-mometum tensor of a generic
perfect fluid is $\w{T} = (\rho + p)\w{u} \otimes \w{u} + p \w{g}$, where $p$
is the fluid pressure. The expression (\ref{e:lem:T_pressureless}) corresponds thus
to the special case $p=0$.
Inside the matter, we link the coordinates $(\tau,\chi,\th,\ph)$ to the fluid by demanding
that they are \defin{comoving}\index{comoving!coordinates} with the fluid, i.e. that a fluid particle
stays at fixed values of $(\chi,\th,\ph)$.
Because the 4-velocity obeys $u^\alpha = \D x^\alpha/\D \tau_{\rm fl}$, where
$\tau_{\rm fl}$ is the fluid proper time [cf. Eq.~(\ref{e:fra:def_u})], this
amounts to set $u^\chi=u^\th = u^\ph = 0$, i.e. to have
\be \label{e:lem:u_par_tau}
    \w{u} = \wpar_\tau .
\ee
A priori, one should have only $\w{u} =  u^\tau\wpar_\tau$, but the
synchronous coordinate condition $g_{\tau\tau} = -1$ along with the
normalization $\w{g}(\w{u},\w{u})=-1$ implies $u^\tau=1$. Since $u^\tau = \D\tau / \D\tau_{\rm fl}$, we get $\tau = \tau_{\rm fl}$ (up to some additive constant), which provides the physical
interpretation of Lemaître coordinate $\tau$ as the \emph{fluid proper time}.

\subsection{Geodesic matter flow}

The equation of energy-momentum conservation $\wnab\cdot\vw{T} = 0$
[Eq.~(\ref{e:fra:divT})], which
follows from the Einstein equation (\ref{e:fra:Einstein_eq}) and the contracted
Bianchi identity (\ref{e:bas:Bianchi_contr}) (cf. Sec.~\ref{s:fra:Einstein_eq}),
implies that
\begin{greybox}
The worldlines of the fluid particles obeying the pressureless matter
model (\ref{e:lem:T_pressureless}) are timelike geodesics of
$(\M,\w{g})$.
\end{greybox}
\begin{proof}
If we plug the energy-momentum tensor (\ref{e:lem:T_pressureless}) in the
energy-momentum conservation law (\ref{e:fra:divT}), we obtain
\[
    \nabla_\mu (\rho u^\mu u^\alpha )  = 0 ,
\]
i.e.
\be \label{e:lem:divT_pressureless}
    \nabla_\mu (\rho u^\mu) u^\alpha + \rho u^\mu \nabla_\mu u^\alpha = 0 .
\ee
Now the two terms in the left-hand side of this equation are orthogonal
to each other, as an immediate consequence of the normalization of the
4-velocity $\w{u}$ [Eq.~(\ref{e:fra:u_unit})]:
$\w{u}\cdot \wnab_{\w{u}}\w{u} = 0$. In particular, $\w{u}$
is a timelike vector, while the 4-acceleration $\wnab_{\w{u}}\w{u}$
is a spacelike one. The only way for Eq.~(\ref{e:lem:divT_pressureless})
to hold is thus that each term
in the left-hand side vanishes separately:
\[
    \nabla_\mu (\rho u^\mu) = 0 \qquad\mbox{and}\qquad u^\mu \nabla_\mu u^\alpha = 0 .
\]
The second equation above is nothing but the geodesic equation [Eq.~(\ref{e:geo:geod_eq_v})]
for the field lines of $\w{u}$, i.e. the fluid worldlines.
\end{proof}
Each fluid particle is thus in free-fall and moves independently of its
neighbours, which is not surprising since the pressure is zero.
This justify the term \defin{dust}\index{dust} given to the matter
model (\ref{e:lem:T_pressureless}).

\subsection{From the Einstein equation to the Lemaître-Tolman system}

Let us write the Einstein equation (\ref{e:fra:Einstein_eq})
in terms of Lemaître synchronous coordinates $(\tau,\chi,\th,\ph)$
and with the energy-momentum tensor (\ref{e:lem:T_pressureless})-(\ref{e:lem:u_par_tau})
in its right-hand side.
As detailed in the notebook~\ref{s:sam:Lemaitre-Tolman},
if one disregards the peculiar case\footnote{For $\Lambda=0$, this case leads to
Datt solution \cite{Datt38}.} $\dert{r}{\chi} = 0$,
the $\tau\chi$ component yields
\be \label{e:lem:a_f_dr}
    a(\tau,\chi) = \frac{1}{f(\chi)} \der{r}{\chi} ,
\ee
where $f(\chi)$ is an arbitrary function of $\chi$.
Accordingly, we may rewrite the metric (\ref{e:lem:metric_sync_coord}) as
\be \label{e:lem:metric_Lemaitre}
    \encadre{ \w{g} = - \dd\tau^2
        + \frac{1}{f(\chi)^2} \left( \der{r}{\chi} \right)^2 \dd\chi^2
        + r(\tau,\chi)^2 \left( \dd\th^2 + \sin^2\th\, \dd\ph^2 \right) } .
\ee

Taking into account (\ref{e:lem:a_f_dr}), the $\chi\chi$ and $\tau\tau$ components of the Einstein equation
yield respectively to (cf. the notebook~\ref{s:sam:Lemaitre-Tolman})
\begin{subequations}\label{e:lem:LTeqs}
\begin{align}
 & \encadre{\left( \der{r}{\tau} \right) ^2 = f(\chi)^2 - 1 + \frac{2m(\chi)}{r(\tau,\chi)}
   + \frac{\Lambda}{3} r(\tau,\chi)^2 } \label{e:lem:LTeqs1} \\
 & \encadre{\frac{\D m}{\D\chi} = 4\pi r(\tau,\chi)^2 \rho(\tau,\chi) \der{r}{\chi} } ,
    \label{e:lem:LTeqs2}
 \end{align}
\end{subequations}
where $m(\chi)$ is another arbitrary function of $\chi$.
There is no other independent component of Einstein equation.
Equations~(\ref{e:lem:LTeqs}) constitute the
\defin{Lemaître-Tolman system}\index{Lemaitre-Tolman system@Lemaître-Tolman system}.

The function $m(\chi)$ is known in the literature as the \defin{Misner-Sharp mass}\index{Misner-Sharp!mass} or \defin{Misner-Sharp energy}\index{Misner-Sharp!energy}, in reference
of a study by Misner and Sharp in 1964 \cite{MisneS64}, despite it has been introduced
more than 30 years earlier by Lemaître \cite{Lemai32}. This quantity is
invariantly defined for any spherically symmetric spacetime from the areal radius $r$:
\be \label{e:lem:def_Misner_Sharp}
    m  := \frac{r}{2} \left( 1 - \nabla_\mu r \nabla^\mu r  - \frac{\Lambda}{3} r^2\right) .
\ee
It is easy to check that the above relation holds in the present case:
we have, thanks to (\ref{e:lem:metric_sync_coord}),
\[
    \nabla_\mu r \nabla^\mu r = g^{\mu\nu} \der{r}{x^\mu} \der{r}{x^\nu}
        = g^{\tau\tau} \left( \der{r}{\tau} \right)^2
        + g^{\chi\chi} \left( \der{r}{\chi} \right)^2
        = - \left( \der{r}{\tau} \right)^2 + \frac{1}{a(\tau,\chi)^2} \left( \der{r}{\chi} \right)^2
\]
Using Eq.~(\ref{e:lem:a_f_dr}), this expression reduces to
\[
    \nabla_\mu r \nabla^\mu r = - \left( \der{r}{\tau} \right)^2 + f(\chi)^2 .
\]
In view of the Lemaître-Tolman equation (\ref{e:lem:LTeqs1}), we conclude that
(\ref{e:lem:def_Misner_Sharp}) holds.

\begin{hist}
The Lemaître-Tolman system (\ref{e:lem:LTeqs}) has been first derived
in 1932 by Georges Lemaître\index{Lemaitre, G.@Lemaître, G.} \cite{Lemai32}:
Eqs.~(\ref{e:lem:metric_Lemaitre}), (\ref{e:lem:LTeqs1}) and (\ref{e:lem:LTeqs2}) are
respectively Eqs.~(8.1), (8.2) and (8.3) of Ref.~\cite{Lemai32}, up to some slight
change of notations.
It however became known as \emph{Tolman model}\index{Tolman model}
or \emph{Tolman-Bondi model}\index{Tolman-Bondi model}, in reference
to posterior works by Richard Tolman\index{Tolman, R.C.} (1934) \cite{Tolma34}
and by Hermann Bondi\index{Bondi, H.} (1947) \cite{Bondi47}.
This happened despite Tolman fully acknowledged Lemaître's work \cite{Lemai32} in his
article \cite{Tolma34} (Tolman actually met Lemaître in 1932-33 during
the latter's trip to United States \cite{Eisen93}) and Bondi \cite{Bondi47} mentioned
that \emph{``Lemaître studies a problem very closely related to ours
and many equations given in the appendix can be found in the} (Lemaître's) \emph{paper''}.
We refer to Eisenstaedt's article \cite{Eisen93} for a detailed historical
study of Lemaître paper \cite{Lemai32} (see also Krasi\'nski's note \cite{Krasi97}).
We follow the suggestion of Pleba\'nski \& Krasi\'nski \cite{PlebaK06}
to call the system (\ref{e:lem:LTeqs}) \emph{Lemaître-Tolman}, and not
merely \emph{Lemaître}, in order to distiguish it from other Lemaître contributions
to general relativity and cosmology.
\end{hist}


\subsection{Solutions for a vanishing cosmological constant} \label{s:lem:sol_lambda_zero}

In the remaining of this chapter, we assume $\Lambda=0$, since we are mainly
interested in gravitational collapse in asymptotically flat spacetimes.
The Lemaître-Tolman equation (\ref{e:lem:LTeqs1}) can be then rewritten
as
\be \label{e:lem:1dim_mechanical}
    \frac{1}{2} \dot{r}^2 - \frac{m(\chi)}{r} = E(\chi) ,
\ee
where $\dot{r} := \partial r /\partial\tau$ and
\be \label{e:lem:E_f_chi}
    E(\chi) := \frac{f(\chi)^2-1}{2} .
\ee
For a fixed value of $\chi$, we recognize in (\ref{e:lem:1dim_mechanical})
the equation ruling the 1-dimensional non-relativistic motion of a
particle in a Newtonian
potential $V=-m/r$; $E(\chi)$ is then nothing but the total
mechanical energy of the particle per unit mass.
As it is well known, the solution of (\ref{e:lem:1dim_mechanical})
depends on the sign of $E(\chi)$:
\begin{itemize}
\item if $E(\chi)>0$, the solution is given in parameterized form (parameter $\eta$) by
\be \label{e:lem:sol_E_pos}
    \left\{ \begin{array}{l}
    \displaystyle\tau = \frac{m(\chi)}{(2E(\chi))^{3/2}} \left( \sinh\eta - \eta \right)
        + \tau_0(\chi) \\[2ex]
    \displaystyle r(\tau,\chi) = \frac{m(\chi)}{2E(\chi)} \left( \cosh\eta - 1 \right)
    \end{array} \right.
\ee
\item if $E(\chi)=0$, the solution is
\be \label{e:lem:sol_E_zero}
    r(\tau,\chi) =  \left( \frac{9 m(\chi)}{2} (\tau -\tau_0(\chi))^2 \right) ^{1/3}
\ee
\item if $E(\chi)<0$, the solution is given in parameterized form (parameter $\eta$) by
\be \label{e:lem:sol_E_neg}
    \left\{ \begin{array}{l}
    \displaystyle\tau =  \frac{m(\chi)}{|2E(\chi)| ^{3/2}} \left( \eta + \sin\eta \right)
    + \tau_0(\chi)  \\[2ex]
    \displaystyle r(\tau,\chi) = \frac{m(\chi)}{|2E(\chi)|} \left( 1 + \cos\eta \right)
    \end{array} \right.
\ee
\end{itemize}
In the above formulas, $\tau_0(\chi)$ is an arbitrary function of $\chi$.
For $E>0$ and $E=0$, it sets the value of $\tau$ for which $r=0$, while
for $E<0$, it sets the value of $\tau$ for which $r$ takes its maximal value
($m/|E|$).

\noindent\emph{Exercise:} prove that each of formulas (\ref{e:lem:sol_E_pos})-(\ref{e:lem:sol_E_neg}) provides
a solution of Eq.~(\ref{e:lem:1dim_mechanical}).

The procedure to get a full solution is (i) choose arbitrary functions
$f(\chi)$, $m(\chi)$ and $\tau_0(\chi)$;
(ii) evaluate $E(\chi)$ via (\ref{e:lem:E_f_chi}); (iii) depending of
on the value of $E(\chi)$, use (\ref{e:lem:sol_E_pos}), (\ref{e:lem:sol_E_zero})
or (\ref{e:lem:sol_E_neg}) to get the solution for $r(\tau,\chi)$;
(iv) plug this solution into the remaining Lemaître-Tolman equation,
Eq.~(\ref{e:lem:LTeqs2}), to get $\rho(\tau,\chi)$ and into
(\ref{e:lem:metric_Lemaitre}) to get the metric tensor.

\subsection{Schwarzschild solution in Lemaître coordinates} \label{s:lem:Schwarzschild}

One can recover Schwarzschild solution from the above setting by
considering the vacuum case, i.e. $\rho=0$. Then
Eq.~(\ref{e:lem:LTeqs2}) imposes $m(\chi)$ to be a constant, which
we shall denote simply by $m$. Regarding the function $f(\chi)$, let us
choose for simplicity $f(\chi)=1$. Then $E(\chi)=0$ and $r(\tau,\chi)$
is given by Eq.~(\ref{e:lem:sol_E_zero}). Since $m$ is constant, we
cannot choose $\tau_0(\chi)$ to be a constant, otherwise
$\partial r/\partial \chi$ would be zero and the metric
(\ref{e:lem:metric_Lemaitre}) would be degenerate.
The simplest non-constant choice is
$\tau_0(\chi) = \chi$. To summarize, the
three functions of $\chi$ determining the solution are set to
\be
    f(\chi) = 1, \qquad m(\chi)=m=\mathrm{const} \qand \tau_0(\chi) = \chi .
\ee
Equation~(\ref{e:lem:sol_E_zero}), with the above values for $m(\chi)$ and $\tau_0(\chi)$,
yields
\be \label{e:lem:r_tau_chi_Schwarz}
    \encadre{ r(\tau,\chi) =  \left( \frac{9m}{2} \right)^{1/3} ( \chi -\tau )^{2/3} }.
\ee
In what follows, we assume $\chi\geq\tau$. Then
\be \label{e:lem:chi_tau_r_Schwarz}
    \chi - \tau = \frac{1}{3} \sqrt{\frac{2}{m}}\,  r^{3/2}
\ee
and
\be \label{e:lem:drdchi_Schwarz}
    \der{r}{\chi} = \left( \frac{4m}{3} \right)^{1/3} (\chi-\tau)^{-1/3}
        = \sqrt{\frac{2m}{r}} .
\ee
Accordingly, Eq.~(\ref{e:lem:metric_Lemaitre}) becomes
\be \label{e:lem:Sch_met_Lem}
    \encadre{  \w{g} = - \dd\tau^2
        + \frac{2m}{r} \dd\chi^2
        + r^2 \left( \dd\th^2 + \sin^2\th\, \dd\ph^2 \right)  } .
\ee
In this expression, $r$ is the function of $(\tau,\chi)$ given
by (\ref{e:lem:r_tau_chi_Schwarz}).

The metric (\ref{e:lem:Sch_met_Lem}) is actually the
Schwarzschild metric of mass parameter $m$. To prove it, let us first
promote $r$ as a coordinate, instead of $\chi$, i.e. we consider the
coordinate system $({x'}^\alpha):=(\tau,r,\th,\ph)$, which are called
\defin{Painlevé-Gullstrand coordinates}\index{Painlevé-Gullstrand coordinates}.
The relation to Lemaître coordinates
$({x}^\alpha)=(\tau,\chi,\th,\ph)$ is obtained by differentiating (\ref{e:lem:r_tau_chi_Schwarz}):
we have clearly $\dert{r}{\tau} = - \dert{r}{\chi}$, so that, taking
into account (\ref{e:lem:drdchi_Schwarz}),
\[
    \dd r =  \sqrt{\frac{2m}{r}} (\dd\chi - \dd\tau) .
\]
Hence
\[
  \sqrt{\frac{2m}{r}}  \dd\chi = \sqrt{\frac{2m}{r}}  \dd\tau + \dd r
\quad \Longrightarrow \quad \frac{2m}{r} \dd\chi^2 = \frac{2m}{r} \dd\tau^2
    + 2 \sqrt{\frac{2m}{r}}  \dd\tau \, \dd r + \dd r^2 .
\]
Substituting this relation in Eq.~(\ref{e:lem:Sch_met_Lem}) yields
the expression of the metric tensor in terms of Painlevé-Gullstrand coordinates:
\be \label{e:lem:metric_PG}
   \encadre{ \w{g} =
     - \left( 1 - \frac{2m}{r} \right)
        \dd\tau^2 + 2 \sqrt{\frac{2m}{r}}  \dd\tau \, \dd r + \dd r^2
        + r^2 \left( \dd\th^2 + \sin^2\th\, \dd\ph^2 \right) } .
\ee
We can rearrange it as
\bea
  \w{g} & = &  - \left( 1 - \frac{2m}{r} \right) \left( \dd\tau^2 - 2
        \frac{\sqrt{\frac{2m}{r}}}{1 - \frac{2m}{r}} \, \dd\tau \, \dd r \right)
            + \dd r^2 + r^2 \left( \dd\th^2 + \sin^2\th\, \dd\ph^2 \right) \nonumber \\
    & = & - \left( 1 - \frac{2m}{r} \right) \left( \dd\tau -
        \frac{\sqrt{\frac{2m}{r}}}{1 - \frac{2m}{r}} \, \dd r \right) ^2
            + \frac{\dd r^2}{1- \frac{2m}{r}}
            + r^2 \left( \dd\th^2 + \sin^2\th\, \dd\ph^2 \right) .  \label{e:lem:met_tau_r}
\eea
If we introduce, instead of $\tau$, a coordinate $t$ such that
\be \label{e:lem:dt_dtau_dr}
    \dd t = \dd\tau -
        \frac{\sqrt{\frac{2m}{r}}}{1 - \frac{2m}{r}} \, \dd r ,
\ee
Eq.~(\ref{e:lem:met_tau_r}) yields immediately the familiar expression
of Schwarzschild metric in Schwarzschild-Droste coordinates $(t,r,\th,\ph)$
[Eq.~(\ref{e:sch:Schwarz_metric_SD})]. Hence this proves that
the vacuum solution (\ref{e:lem:Sch_met_Lem}) is nothing but
Schwarzschild metric. Incidently, since our starting point was
the most general metric for a spherically symmetric spacetime
[Eq.~(\ref{e:lem:metric_sync_coord})], we have proven
\defin{Birkhoff's theorem}\index{Birkhoff's theorem}:
\begin{greybox}
In vacuum, the unique spherically symmetric solution of the 4-dimensional
Einstein equation with $\Lambda=0$ is Schwarzschild metric.
\end{greybox}
In particular, outside any spherically symmetric body, the spacetime
is a piece of Schwarz\-schild spacetime. Note that this implies that this part of
spacetime is static, even if the central body is not (for instance oscillate
radially, keeping its spherical symmetry). In other words, there are no
gravitational waves in spherical symmetry.
\begin{remark}
Birkhoff's theorem can be viewed as a generalization of Gauss' theorem in Newtonian gravity:
the gravitational field outside any spherical source is entirely determined by the mass $m$ of
the source, being identical to that generated by a point of mass $m$ located at the symmetry
center.
\end{remark}

The relation between Lemaître coordinates and
Schwarzschild-Droste ones can be made explicit by integrating
Eq.~(\ref{e:lem:dt_dtau_dr}); one gets
\be
    \tau = t + 4m \sqrt{\frac{r}{2m}} + 2 m \ln \left(
        \frac{\sqrt{r/2m} - 1}{\sqrt{r/2m} + 1} \right) + \mathrm{const}.
\ee
The expression of $\chi$ in terms of $(t,r)$ is then deduced from
Eq.~(\ref{e:lem:chi_tau_r_Schwarz}):
\be
    \chi = t + 4m \sqrt{\frac{r}{2m}} \left( 1 + \frac{r}{6m} \right)
        + 2 m \ln \left(
        \frac{\sqrt{r/2m} - 1}{\sqrt{r/2m} + 1} \right)  + \mathrm{const}.
\ee
We deduce easily from these formulas the expression of the stationarity
Killing vector $\w{\xi}$ of Schwarzschild spacetime in terms of the
Lemaître coordinates. Since $\w{\xi} = \wpar_t$ [Eq.~(\ref{e:sch:xi_wpar_t})],
and the above formulas
imply $\dert{\tau}{t} =1$ and $\dert{\chi}{t}=1$, we get, applying
the chain rule $\dert{}{t} = \dert{}{\tau} \times \dert{\tau}{t} + \dert{}{\chi} \times \dert{\chi}{t}$,
\be \label{e:lem:Killing_vect_Schwarz}
    \encadre{\w{\xi} = \wpar_\tau + \wpar_\chi }.
\ee
\begin{remark}
Although very simple, this relation shows that Lemaître coordinates are \emph{not}
adapted to the spacetime symmetry generated by the Killing vector $\w{\xi}$:
the latter does not coincide with any Lemaître coordinate vector. This reflects
the fact that the metric components (\ref{e:lem:Sch_met_Lem}) depend on $\tau$
(via the function $r(\tau,\chi)$), in addition to $\chi$.
\end{remark}

Despite the vacuum hypothesis implies that we can no longer interpret
Lemaître coordinates as comoving with some free-falling dust as in Sec.~\ref{s:lem:hyp},
their geodesic character remains. Indeed, the vector $\w{u} := \wpar_\tau$ is geodesic:
$\wnab_{\w{u}}\w{u} = 0$, which implies that the curves $(\chi,\th,\ph)=\mathrm{const}$
are timelike geodesics. Moreover, the conserved energy per unit mass along these geodesics
(cf. Sec.~\ref{s:geo:sym}) is
\[
     \varepsilon = - \w{\xi}\cdot\w{u} = - (\wpar_\tau + \wpar_\chi)\cdot \wpar_\tau
        = - \underbrace{g_{\tau\tau}}_{-1} - \underbrace{g_{\chi\tau}}_{0} = 1 ,
\]
where use has been made of (\ref{e:lem:Killing_vect_Schwarz}).
$\varepsilon=1$ means
that the geodesics are marginally bound\index{marginally!bound!geodesic}: they
describe a free fall from rest at infinity.

As it is clear on the metric components (\ref{e:lem:Sch_met_Lem}),
a key feature of Lemaître coordinates is to be regular at $r=2m$, i.e.
accross the event horizon of Schwarzschild spacetime, contrary to the
Schwarzschild-Droste coordinates.

\begin{hist}
The Schwarzschild metric in the form (\ref{e:lem:Sch_met_Lem}) has
been obtained in 1932 by Georges Lemaître\index{Lemaitre, G.@Lemaître, G.} \cite{Lemai32},
as a vacuum solution of the Lemaître-Tolman system: cf. Eq.~(11.12) of Ref.~\cite{Lemai32}.
Remarkably, Lemaître pointed out that the metric components (\ref{e:lem:Sch_met_Lem}) are
regular at $r=2m$ and was the first author to conclude that the singularity of
Schwarzschild's solution at $r=2m$ is a mere coordinate singularity.
As pointed out in the historical note on p.~\pageref{n:sch:Eddington_coord}, eight years before,
Eddington exhibited a coordinate system that is regular at $r=2m$ \cite{Eddin1924} but he
did not mention this feature.
\end{hist}

\subsubsection{Painlevé-Gullstrand coordinates on Schwarzschild spacetime}

Painlevé-Gullstrand coordinates\index{Painlevé-Gullstrand coordinates} $(\tau,r,\th,\ph)$,
which have been introduced
in our way from Lemaître coordinates to Schwarzschild-Droste ones,
have a noticeable feature: the hypersurfaces $\tau=\mathrm{const}$ are
flat 3-manifolds, i.e. the metric induced of them by $\w{g}$ is the flat Euclidean metric.
This is immediate if we set $\dd\tau=0$ in Eq.~(\ref{e:lem:metric_PG}):
\[
    \w{g}^{\tau=\mathrm{const}} = \dd r^2 + r^2 \left( \dd\th^2 + \sin^2\th\, \dd\ph^2 \right) ,
\]
which is nothing but the Euclidean metric expressed in spherical coordinates
$(r,\th,\ph)$. This proves that Schwarzschild spacetime can be sliced
by a family of flat hypersurfaces. The associated 3+1 decomposition
of the metric is revealed by rewritting (\ref{e:lem:metric_PG})
as
\be
       \encadre{ \w{g} =
       - \dd\tau^2 + \left( \dd r + \sqrt{\frac{2m}{r}} \dd\tau \right)^2
        + r^2 \left( \dd\th^2 + \sin^2\th\, \dd\ph^2 \right) } .
\ee
One reads on this expression that the \emph{lapse function}\index{lapse function}
(see e.g. Ref.~\cite{Gourg12}) is $N=1$ and that the \emph{shift vector}\index{shift vector}
is $\beta^i = (\sqrt{r/2m},0,0)$. Finding a lapse function equal to one reflects simply
that the coordinate time $\tau$ is some observer proper time: that of the
marginally bound radial geodesics discussed above.

Another interesting property of Painlevé-Gullstrand coordinates, which they
share with Lemaître ones, is to be regular at $r=2m$: despite the vanishing of
$g_{\tau\tau}$ there, as read on (\ref{e:lem:metric_PG}), the determinant
of the metric components (\ref{e:lem:metric_PG}) is not vanishing,
thanks to the off-diagonal term $g_{\tau r}$. Indeed, $\det\left( g_{\alpha\beta}\right) = -r^4\sin^2\th$, which is clearly nonzero, except on the
axis $\th=0$ or $\pi$.

\begin{remark}
Painlevé-Gullstrand coordinates $(\tau,r,\th,\ph)$ can be seen as a timelike analog of the
ingoing
Eddington-Finkelstein (IEF) coordinates $(\tilde{t},r,\th,\ph)$ introduced in Sec.~\ref{s:sch:EF_coord}. Both coordinate
systems are based on ingoing radial geodesics of Schwarzschild spacetime, these
geodesics being null for IEF and timelike for Painlevé-Gullstrand.
\end{remark}

The reader is referred to Ref.~\cite{MarteP01} for a detailed
discussion of Painlevé-Gullstrand coordinates.

\begin{hist}
Painlevé-Gullstrand coordinates have been introduced in 1921 by the
French mathematician
Paul Painlevé\index{Painlevé, P.} \cite{Painl1921}, as well as by the Swedish physicist and ophtalmologist
Allvar Gullstrand\index{Gullstrand, A.} (1911 laureate of the Nobel Prize in Medicine) in 1922 \cite{Gulls1922}.
\end{hist}

\section{Oppenheimer-Snyder collapse}

\subsection{Pressureless collapse of a star from rest}

Let us consider
a spherical star in hydrostatic equilibrium that suddenly looses any
pressure support to counterbalance gravity. This roughly models the astrophysical
scenario of gravitational collapse\index{gravitational!collapse}
of the iron core of a massive star at the
end of thermonuclear evolution, a phenomon which can ultimately gives birth
to a supernova\index{supernova} if some bounce occurs. In this astrophysical event, the
pressure never vanishes, but after the collapse has reached a certain stage, it
plays a negligle dynamical role, so that the pressureless (dust) approximation
developed in this chapter is quite good.

Given the assumed spherical symmetry, the spacetime metric outside the star
is Schwarzschild metric, by virtue of Birkhoff's theorem (cf. Sec.~\ref{s:lem:Schwarzschild}).
Let us then characterize the star by its gravitational mass $m_0$ and its
areal radius\footnote{Cf. Sec.~\ref{s:sch:static_spher}.} $r_0$ at the start of the collapse; $m_0$
is nothing but the mass parameter of the Schwarzschild metric. It stays
constant during all the collapse.
We shall describe the spacetime exterior
to the star by means the ingoing Eddington-Finkelstein coordinates (IEF)
$(\ti, r, \th,\ph)$
introduced in Sec.~\ref{s:sch:EF_coord}, because they are regular on the
black hole event horizon, contrary to Schwarzschild-Droste coordinates.
The exterior metric is then given by Eq.~(\ref{e:sch:Schwarz_metric_EF}) with
$m=m_0$:
\be \label{e:lem:OS:exterior_metric}
    \w{g} =
            -\left( 1 - \frac{2 m_0}{r} \right)\, \dd \ti^2
            + \frac{4m_0}{r} \, \dd \ti \, \dd r
            + \left( 1 + \frac{2 m_0}{r} \right)\, \dd r^2
        + r^2 \left( \dd\th^2 + \sin^2\th\, \dd\ph^2 \right) .
\ee

Let us denote by $r_{\rm s}(\tau)$ the areal radius of the star, i.e. the coordinate $r$
of its surface, as a function of the proper time $\tau$ of a matter particle
at the surface of the star. If the origin of $\tau$ is set to
the start of the collapse, we have thus
\be
    r_{\rm s}(0) = r_0.
\ee
Due to the pressureless hypothesis, the surface particles follow timelike radial geodesics
of Schwarzschild spacetime, as studied in Sec.~\ref{s:ges:radial_free_fall}.
Since the collapse starts from rest, the function $r_{\rm s}(\tau)$ is
given by the parametric expression (\ref{e:ges:sol_radial_infall}):
\be \label{e:lem:OS:evol_star_surf}
    \left\{ \begin{array}{l}
    \displaystyle\tau = \sqrt{\frac{r_0^3}{8 m_0}}  \left( \eta + \sin\eta \right) \\[3ex]
    \displaystyle r_{\rm s}(\tau) = \frac{r_0}{2} \left( 1 + \cos\eta \right)
    \end{array} \right.
    \qquad 0 \leq \eta \leq \pi .
\ee
The IEF coordinate time $\ti$ corresponding to the surface proper time $\tau$
is given by Eq.~(\ref{e:ges:sol_radial_infall_ti}):
\be \label{e:lem:OS:ti_star_surf}
     \ti = 2m_0 \left\{ \sqrt{\frac{r_0}{2m_0} - 1} \left[ \eta + \frac{r_0}{4m_0}
    (\eta + \sin\eta) \right]
    +  2 \ln \left[ \cos\frac{\eta}{2}  + \left(\frac{r_0}{2m_0} - 1\right)^{-1/2} \sin \frac{\eta}{2} \right] \right\}  ,
\ee
where $\ti = 0$ is assumed at the start of the collapse.
Note that by combining Eqs.~(\ref{e:lem:OS:evol_star_surf}) and (\ref{e:lem:OS:ti_star_surf}),
one can obtain $r_s = r_s(\ti)$, i.e. the evolution law of the stellar surface
in terms of the IEF coordinate time $\ti$.
The right panel in Fig.~\ref{f:ges:radial_infall} (with $m=m_0$)
depicts it for various values of $m_0/r_0$.


\subsection{Oppenheimer-Snyder solution}

The metric (\ref{e:lem:OS:exterior_metric}) is valid only for $r \geq r_{\rm s}(\ti)$.
The metric inside the star is given by one of the solutions
of the Lemaître-Tolman equations given in Sec.~\ref{s:lem:sol_lambda_zero}.
Actually only the solutions with $E(\chi)<0$, i.e. those given by Eq.~(\ref{e:lem:sol_E_neg}),
are to be considered because they are the only ones allowing a collapse starting
from rest.

As stressed in Sec.~\ref{s:lem:sol_lambda_zero}, formula~(\ref{e:lem:sol_E_neg}),
in conjunction with Eqs.~(\ref{e:lem:LTeqs2}) and
(\ref{e:lem:metric_Lemaitre}),
defines an infinity of solutions depending on the functions
$f(\chi)$, $m(\chi)$ and $\tau_0(\chi)$, which can be chosen arbitrarily.
We shall actually focus on the simplest of these solutions, which is that
describing a \emph{homogeneous} star, i.e. an object whose density
$\rho$ in the matter frame, as defined by Eq.~(\ref{e:lem:T_pressureless}),
is constant at any given proper time $\tau$. In other words, $\rho$, which
is a priori a function of $(\tau,\chi)$, is actually a function of $\tau$ only.
This is certainly a crude approximation of a real star (or stellar iron core),
which has a non-constant density profile.

As we are going to see, the homegeneous dust star collapse is obtained by the
following choice of the free functions:
\be
    f(\chi) = \cos\chi,\qquad
    m(\chi) = \frac{a_0}{2} \sin^3\chi
    \qand \tau_0(\chi) = 0 ,
\ee
with
\be
    a_0 := \sqrt{\frac{r_0^3}{2 m_0}} .
\ee




\subsection{Black hole formation}




















