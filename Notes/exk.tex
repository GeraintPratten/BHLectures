\chapter{Extremal Kerr black hole}
\label{s:exk}

\minitoc

\section{Introduction}

The Kerr solution of Einstein equation has been introduced in Sec.~\ref{s:ker:Kerr_solution};
it depends on two parameters: the mass $m > 0$ and the spin parameter
$a \geq 0$.
In Chaps.~\ref{s:ker}--\ref{s:gik}, we have considered the Kerr solution with $0<a<m$,
while the case $a=0$ (Schwarzschild solution) has been treated in Chaps.~\ref{s:sch}--\ref{s:max}.
Here we focus on the case $a=m$. As we going to see, it has many properties that are not
shared by the Kerr solution with $a<m$. In particular, the black hole event horizon is
degenerate, in the sense defined in Sec.~\ref{s:neh:classif_KH}: the surface gravity
$\kappa$ vanishes. Note that $a=m$ corresponds to the highest value of $a$
for which the Kerr solution corresponds to a black hole.
Indeed, for $a> m$, the Kerr metric is still an exact
solution of the vacuum Einstein equation, but it describes a \emph{naked singularity}\index{naked singularity} (cf. Sec.~\ref{s:max:naked_sing}):
the ring curvature singularity is not hidden by any black hole horizon to asymptotic observers.


%%%%%%%%%%%%%%%%%%%%%%%%%%%%%%%%%%%%%%%%%%%%%%%%%%%%%%%%%%%%%%%%%%%%%%%%%%%%%%%

\section{Definition and basic properties}

\subsection{The extremal Kerr solution}

Let us consider the manifold $\R^2\times\SS^2$ and describe it by
coordinates $(\ti, r, \th,\tph)$ such that $(\ti,r)$ cover $\R^2$
and $(\th,\tph)$ are standard spherical coordinates on $\SS^2$.
The \defin{extremal Kerr spacetime}\index{extremal!Kerr spacetime}\index{Kerr!extremal -- spacetime}
of mass $m>0$ is defined as the pair $(\M, \w{g})$ where the manifold $\M$ is the following open subset of $\R^2\times\SS^2$:
\be \label{e:exk:def_M}
 \M := \R^2\times\SS^2 \setminus \ring
\ee
with
\be \label{e:exk:def_ring}
    \ring := \left\{ p \in \R^2\times\SS^2,
        \quad r(p) = 0 \ \mbox{and}\ \th(p) = \frac{\pi}{2} \right\} ,
\ee
and the metric $\w{g}$ has the following expression in terms of the coordinates
$(x^{\tilde{\alpha}}) = (\ti, r, \th,\tph)$:
\be \label{e:exk:metric_Kerr_3p1}
    \encadre{
    \begin{array}{ll}
    g_{\tilde{\mu}\tilde{\nu}}\, \D x^{\tilde{\mu}} \, \D x^{\tilde{\nu}}   = &
    \displaystyle - \left( 1 - \frac{2m r}{\rho^2} \right)  \D \ti^2
    + \frac{4m r}{\rho^2} \D\ti\, \D r
    - \frac{4 m^2  r \sin^2\th}{\rho^2} \,  \D \ti\, \D\tph \\[2ex]
    &\displaystyle  + \left( 1 + \frac{2m r}{\rho^2} \right) \D r^2
     - 2 m \left( 1 + \frac{2m r}{\rho^2} \right) \sin^2\th \, \D r\, \D \tph \\[2ex]
    & \displaystyle + \rho^2 \D \th^2
    + \left( r^2 + m^2 + \frac{2 m^3 r \sin^2\th}{\rho^2} \right)
    \sin^2\th \, \D \tph^2 ,
    \end{array}
    }
\ee
with
\be
    \rho^2 := r^2 + m^2\cos^2\th .
\ee
In this context, the coordinates $(x^{\tilde{\alpha}}) = (\ti, r, \th,\tph)$
are called the
\defin{3+1 Kerr coordinates}\index{3+1!Kerr coordinates}\index{Kerr!coordinates!3+1 --}
and we recognize in (\ref{e:exk:metric_Kerr_3p1}) the limit $a\to m$ of
expression (\ref{e:ker:metric_Kerr_3p1}) for the Kerr metric with $a< m$
in the 3+1 Kerr coordinates.

The metric (\ref{e:exk:metric_Kerr_3p1}) is regular
in all $\M$, since the components $g_{\tilde{\alpha}\tilde{\beta}}$ are singular only
for $\rho=0$, i.e. for $r=0$ and $\th=\pi/2$, which defines  the set $\ring$ that has precisely been excluded in
the definition (\ref{e:exk:def_M}) of $\M$. The Kretschmann curvature
invariant\index{Kretschmann scalar! of Kerr metric} $K := R_{\mu\nu\rho\sigma} R^{\mu\nu\rho\sigma}$
is given by Eq.~(\ref{e:ker:Kretschmann}) with $a=m$; it diverges for $\rho\to 0$. Therefore, as
for the Kerr spacetime with $a<m$ (cf. Sec.~\ref{s:ker:singularities}), we shall call $\ring$ the \defin{ring singularity}\index{ring!singularity}\index{singularity!ring --}
of the extremal Kerr spacetime. Note that, formally, it is not part of the spacetime manifold
[cf. Eq.~(\ref{e:exk:def_M})].

Moreover, the Ricci tensor of the metric (\ref{e:exk:metric_Kerr_3p1}) is identically zero in all
$\M$ (see e.g. the notebook~\ref{s:sam:Kerr_Kerr_coord}). Hence, we have:
\begin{greybox}
The metric $\w{g}$ of the extremal Kerr spacetime is a solution of Einstein equation\index{Einstein!equation} (\ref{e:fra:Einstein_eq})
in vacuum ($\w{T}=0$) and with a vanishing cosmological constant ($\Lambda=0$).
\end{greybox}

\subsection{Boyer-Lindquist coordinates}

For $a<m$, the Kerr manifold $\M$ has been split in three open regions,
$\M_{\rm I}$, $\M_{\rm II}$ and $\M_{\rm III}$,  separated by the two Killing
horizons $\Hor$ and $\Hor_{\rm in}$
[cf. Eqs.~(\ref{e:ker:def_M_Kerr_spacetime}) and (\ref{e:ker:def_M_BL})].
Since $\Hor$ was defined by $r=r_+:=m + \sqrt{m^2 - a^2}$ [Eq.~(\ref{e:ker:def_H})]
and $\Hor_{\rm in}$
by $r=r_-:=m - \sqrt{m^2 - a^2}$ [Eq.~\ref{e:ker:def_H_in})],
we notice that in the $a\to m$ limit
$r_+ = r_- = m$, so that $\Hor$ and $\Hor_{\rm in}$ coincide and the region
$\M_{\rm II}$, which is bounded by $\Hor$ and $\Hor_{\rm in}$, disappears.
Accordingly, we shall split the extremal Kerr manifold $\M$ in two open regions only,
$\M_{\rm I}$ and $\M_{\rm III}$, separated by a single hypersurface $\Hor$:
\be
    \encadre{\M = \M_{\rm I} \cup \Hor \cup \M_{\rm III} },
\ee
with
\begin{subequations}
\begin{align}
    \M_{\rm I} & := \left\{ p \in \M, \quad r(p) > m \right\} \\
    \Hor & := \left\{ p \in \M, \quad r(p) = m \right\} \\
    \M_{\rm III} & := \left\{ p \in \M, \quad r(p) < m \right\} .
\end{align}
\end{subequations}
\begin{remark}
We are using the notation $\M_{\rm III}$ for the ``second'' region
to be consistent with Chaps.~\ref{s:ker}--\ref{s:gik}, more precisely with
the $a\to m$ limit of the results obtained in these chapters.
\end{remark}
The second order polynomial in $r$ introduced in Chap.~\ref{s:ker},
$\Delta=r^2 - 2 m r + a^2 = (r - r_+)(r - r_-)$ reduces to $\Delta = (r - m)^2$
in the limit $a\to m$. Its double root, $r=m$, defines the hypersurface $\Hor$.

On the region $\M_{\rm BL}:= \M \setminus \Hor = \M_{\rm I} \cup \M_{\rm III}$, one may introduce
the \defin{Boyer-Lindquist coordinates}\index{Boyer-Lindquist coordinates}
$(t,r,\th,\ph)$ such that $(r,\th)$ are the same coordinates as in the
3+1 Kerr coordinates, while $t$ and $\ph$ are related to the 3+1 Kerr coordinates
$\ti$, $r$ and $\tph$ by
\begin{subequations}
\label{e:exk:3p1_Kerr_to_BL}
\begin{align}
    t & = \ti + 2 m \left( \frac{m}{r - m} - \ln \left| \frac{r - m}{m} \right| \right) \\
    \ph & = \tph + \frac{m}{r - m} .
\end{align}
\end{subequations}
The differential versions of these relations are
\begin{subequations}
\label{e:exk:dt_dph}
\begin{align}
    \D t & = \D \ti  - \frac{2m r}{(r- m)^2} \, \D r \\
    \D \ph & = \D \tph - \frac{m}{(r - m)^2 } \, \D r .
\end{align}
\end{subequations}
The metric components $(g_{\alpha\beta})$ with respect to the Boyer-Lindquist coordinates $(x^\alpha) = (t,r,\th,\ph)$ are given by
\be \label{e:exk:metric_BL}
    \encadre{
    \begin{array}{ll}
    g_{\mu\nu}\,  \D x^\mu \D x^\nu  = &
    \displaystyle - \left( 1 - \frac{2m r}{\rho^2} \right) \, \D t^2
    - \frac{4 m^2  r \sin^2\th}{\rho^2} \,  \D t\, \D\ph
    + \frac{\rho^2}{(r-m)^2} \, \D r^2  \\[2ex]
    & \displaystyle + \rho^2 \D \th^2
    + \left( r^2 + m^2 + \frac{2 m^3 r \sin^2\th}{\rho^2} \right)
    \sin^2\th \, \D \ph^2 .
    \end{array}
    }
\ee
This expression can be obtained either by taking the limit $a\to m$ of Eq.~(\ref{e:ker:metric_BL})
of by using (\ref{e:exk:dt_dph}) to substitute $\D\ti$ and $\D\tph$ in Eq.~(\ref{e:exk:metric_Kerr_3p1}).
We note that $g_{rr}\to +\infty$ when $r\to m$, which reflects the singularity of Boyer-Lindquist
coordinates on $\Hor$. This singularity is clearly apparemt in the coordinate transformations
(\ref{e:exk:3p1_Kerr_to_BL}).

\subsection{The degenerate horizon}


\section{Maximal analytic extension}

\section{Near-horizon extremal Kerr metric}

\subsection{The extremal Kerr throat}


