\chapter{Kerr-Schild metrics}
\label{s:ksm}

\minitoc

\section{Generic Kerr-Schild spacetimes}

\subsection{Definition} \label{s:ksm:def_Kerr_Schild}

A spacetime $(\M,\w{g})$ is said to have a
\defin{Kerr-Schild metric}\index{Kerr-Schild!metric} iff the metric tensor $\w{g}$
can be written
\be \label{e:ksm:g_KerrSchild}
   \encadre{ \w{g} = \w{f} + 2 H \uu{k}\otimes\uu{k} },
\ee
or equivalently (in index notation):
\be \label{e:ksm:g_KerrSchild_comp}
    \encadre{ g_{\alpha\beta} = f_{\alpha\beta} + 2 H k_\alpha k_\beta },
\ee
where $\w{f}$ is a flat Lorentzian metric on $\M$ (Minkowski metric),
$H$ is a scalar field on $\M$ and $\uu{k}$ is a 1-form on $\M$ such that the
vector associated to it via $\w{f}$ is a null vector of the metric
$\w{f}$:
\be
     f^{\mu\nu} k_\mu k_\nu = 0 ,
\ee
where $f^{\mu\nu}$ stands for the components of the inverse of the metric
$\w{f}$ (i.e. $f^{\alpha\mu} f_{\mu\beta} = \delta^\alpha_{\ \; \beta}$).

A motivation for studying
Kerr-Schild metrics is that the inverse metric has a simple expression:
\be \label{e:ksm:g_inverse}
    g^{\alpha\beta} = f^{\alpha\beta} - 2 H k^\alpha k^\beta ,
\ee
where
\be \label{e:ksm:k_up_comp}
    k^\alpha := f^{\alpha\mu} k_\mu .
\ee
\emph{Proof:} we have successively:
\bea
   (f^{\alpha\mu} - 2 H k^\alpha k^\mu) g_{\mu\beta} & = &
   (f^{\alpha\mu} - 2 H k^\alpha k^\mu) (f_{\mu\beta} + 2 H k_\mu k_\beta) \nonumber \\
   & = & \underbrace{f^{\alpha\mu} f_{\mu\beta}}_{\delta^\alpha_{\ \; \beta}}
    + 2 H \underbrace{f^{\alpha\mu} k_\mu}_{k^\alpha} k_\beta
    - 2 H k^\alpha \underbrace{k^\mu f_{\mu\beta}}_{k_\beta}
    - 4 H^2 k^\alpha \underbrace{k^\mu k_\mu}_{0} k_\beta \nonumber \\
   & = & \delta^\alpha_{\ \; \beta} , \nonumber
\eea
which establishes Eq.~\eqref{e:ksm:g_inverse}. \qed

Given (\ref{e:ksm:g_inverse}), it is easy to see that the vector field
$\w{k}$ associated
to the 1-form $\uu{k}$ by $\w{g}$-duality (cf. Sec.~\ref{s:bas:metric_dual})
is the same as the vector field
obtained by $\w{f}$-duality:
\[
    g^{\alpha\mu} k_\mu = (f^{\alpha\mu} - 2 H k^\alpha k^\mu) k_\mu
                        = \underbrace{f^{\alpha\mu} k_\mu}_{k^\alpha}
                          - 2H k^\alpha \underbrace{k^\mu k_\mu}_{0}
                        = k^\alpha .
\]
Accordingly, we may write the components of $\w{k}$ simply as $k^\alpha$
without having to specify whether the index has been raised with the
metric $\w{g}$ or with the metric $\w{f}$:
\be
    k^\alpha = f^{\alpha\mu} k_\mu  = g^{\alpha\mu} k_\mu .
\ee
It follows immediately that $\w{k}$ is a null vector field for
both metrics:
\be
   \encadre{ \w{g}(\w{k}, \w{k}) = \w{f}(\w{k}, \w{k}) = 0 }.
\ee


If $(\M,\w{g})$ is a spacetime of Kerr-Schild type, then \defin{Kerr-Schild
coordinates}\index{Kerr-Schild!coordinates}
are coordinates $(x^\alpha) = (t,x,y,z)$ that are Minkowskian\index{Minkowskian!coordinates}
for $\w{f}$, i.e. coordinates in which the components of the flat metric
$\w{f}$ take the form
\be \label{e:ksm:ds_eta}
    f_{\mu\nu} \, \D x^\mu \, \D x^\nu = - \D t^2 + \D x^2 + \D y^2
        + \D z^2 .
\ee

\subsection{Basic property}

\begin{greybox}
Let $\w{g}$ be a Kerr-Schild metric.
If $\w{g}$ obeys the vacuum Einstein
equation\index{Einstein!equation}, i.e. if the Ricci tensor of $\w{g}$ vanishes identically:
\be
    R_{\alpha\beta} = 0 ,
\ee
then the scalar field $H$ appearing in Eq.~(\ref{e:ksm:g_KerrSchild})
can be chosen so that $\w{k}$ is a geodesic vector field\footnote{See Sec.~\ref{s:geo:gener_param} for the definition of a geodesic vector field.}:
\be \label{e:ksm:k_geodesic}
    \encadre{ k^\mu \nabla_\mu k^\alpha = 0 },
\ee
where $\nabla$ stands for the covariant derivative associated with $\w{g}$.
\end{greybox}

The proof of the above proposition can be found in Ref.~\cite{KerrS65}.


%%%%%%%%%%%%%%%%%%%%%%%%%%%%%%%%%%%%%%%%%%%%%%%%%%%%%%%%%%%%%%%%%%%%%%%%%%%%%%%

\section{Case of Kerr spacetime}

\subsection{Kerr-Schild form}

Let consider the Kerr spacetime $(\M,\w{g})$, where $\M$ is the manifold
(\ref{e:ker:def_M_Kerr_spacetime}): $\M = \R^2\times\SS^2 \setminus \ring$
and $\w{g}$ is the metric tensor given by Eq.~(\ref{e:ker:metric_Kerr_3p1})
in terms of the 3+1 Kerr coordinates $(\ti, r, \th,\tph)$ introduced in
Sec.~\ref{s:ker:3p1_Kerr_coord}.
Let us show that $\w{g}$ is a Kerr-Schild metric, with the associated null vector
field $\w{k}$ being nothing but the vector field generating the principal
ingoing null geodesics $\Li^{\rm in}_{(v,\th,\tph)}$  discussed in Sec.~\ref{s:ker:principal_geod}.
Its expression in terms of the 3+1 Kerr coordinates is given by Eq.~(\ref{e:ker:k_ti_tr}):
\be \label{e:ksm:k_Kerr}
    \encadre{ \w{k} = \wpar_\ti - \wpar_{\tilde r} }.
\ee
In other words, the components of $\w{k}$ with respect to the 3+1 Kerr coordinates $(\ti, r, \th,\tph)$ are $k^\alpha = (1, -1, 0, 0)$.
The 1-form $\uu{k}$ associated to $\w{k}$ by $\w{g}$-duality is easily computed
from $k_\alpha = g_{\alpha\mu} k^\mu$, with $g_{\alpha\mu}$ given by
Eq.~(\ref{e:ker:metric_Kerr_3p1}). We get
$k_\alpha = (-1, -1, 0, a\sin^2\th)$, i.e.
\be \label{e:ksm:k_form_Kerr}
    \uu{k} = - \dd \ti - \dd r + a\sin^2\th \, \dd\tph .
\ee
Let us then introduce the symmetric bilinear form
\be
    \w{f} := \w{g} - 2 H \uu{k} \otimes \uu{k} ,
\ee
where $H$ is the following scalar field on $\M$:
\be \label{e:ksm:H_Kerr}
   \encadre{ H := \frac{m r}{\rho^2} },
\ee
with $\rho^2 := r^2 + a^2 \cos^2\th$ [Eq.~(\ref{e:ker:def_rho2})].
The expression of $\w{f}$ in terms of the 3+1 Kerr coordinates is deduced
from that of $\w{g}$ [Eq.~(\ref{e:ker:metric_Kerr_3p1})] and that of
$\uu{k}$ [Eq.~(\ref{e:ksm:k_form_Kerr})]:
\be \label{e:ksm:f_Kerr}
  \encadre{  f_{\mu\nu}\,  \D x^\mu \D x^\nu = - \D\ti^2 + \D r^2 - 2 a \sin^2\th \, \D r \, \D\tph
    + \rho^2\, \D \th^2 + (r^2 + a^2)\sin^2\th\, \D \tph^2 }.
\ee
It is easy to check that $f^{\alpha\beta} := g^{\alpha\beta} + 2 H k^\alpha k^\beta$
defines an inverse of $\w{f}$: $f^{\alpha\mu} f_{\mu\beta} = \delta^\alpha_{\ \; \beta}$
(computation similar to that in Sec.~\ref{s:ksm:def_Kerr_Schild}). Hence the
symmetric bilinear form $\w{f}$ is nondegenerate; this implies that $\w{f}$
is a \emph{metric tensor} on $\M$ (cf. Sec.~\ref{s:bas:metric}).
Given the components (\ref{e:ksm:k_form_Kerr}), it is immediate to check that $\w{k}$ is a null vector for $\w{f}$ as well: $\w{f}(\w{k}, \w{k}) = 0$.
Moreover, $\w{f}$ is a \emph{flat} metric, since a direct computation of its
Riemann tensor (cf. the notebook \ref{s:sam:Kerr_Schild}) reveals that
\be
    \mathrm{\bf Riem}(\w{f}) = 0 .
\ee
In view of the definition given in Sec.~\ref{s:ksm:def_Kerr_Schild},
we conclude that
\begin{greybox}
The Kerr metric $\w{g}$ is a Kerr-Schild metric, i.e. it can be written in
the form (\ref{e:ksm:g_KerrSchild}) with the flat metric
$\w{f}$ given by Eq.~(\ref{e:ksm:f_Kerr}), the scalar field $H$ given
by Eq.~(\ref{e:ksm:H_Kerr}) and the null vector $\w{k}$ given by
Eq.~(\ref{e:ksm:k_Kerr}), $\w{k}$ being the tangent vector field to
the principal ingoing null geodesics.
\end{greybox}
In Sec.~\ref{s:ker:principal_geod},
we have already noticed that $\w{k}$ is a geodesic vector:
$\wnab_{\w{k}}\, \w{k} = 0$ [Eq.~(\ref{e:ker:nab_k_k})], in agreement with
(\ref{e:ksm:k_geodesic}).

\begin{remark}
The Kerr metric can also be brought to the Kerr-Schild form by using the
tangent vector field to the principal \emph{outgoing} null geodesics. Hence
the Kerr-Schild decomposition (\ref{e:ksm:g_KerrSchild}) is not unique for
the Kerr metric.
\end{remark}

\subsection{Kerr-Schild coordinates}

It is not immediately obvious that the metric $\w{f}$ given by
Eq.~(\ref{e:ksm:f_Kerr}) is a flat Lorentzian metric.
Let us introduce coordinates in which $\w{f}$ takes a manifestly Minkowskian
form, i.e. Kerr-Schild coordinates, according to the nomenclature introduced
in Sec.~\ref{s:ksm:def_Kerr_Schild}.


The Kerr spacetime (cf. Chap.~\ref{s:ker}) is of Kerr-Schild type
and one can set Kerr-Schild coordinates\footnote{The first coordinate
is denoted by $\ti$ to distinguish it from the Boyer-Lindquist coordinate $t$,
see below.} $(\ti, x, y, z)$ on it with
\be \label{e:ksm:H_Kerr}
   \encadre{ H = \frac{m r^3}{r^4 + a^2 z^2} }
\ee
and
\be \label{e:ksm:k_a_Kerr}
    \encadre{ k_\alpha = \left( 1,\ \frac{r x + a y}{r^2 + a^2},\
        \frac{r y - a x}{r^2 + a^2},\ \frac{z}{r} \right) },
\ee
where $m$ and $a$ are respectively the mass and specific angular momentum
parameters of the Kerr spacetime and
$r$ is the following function of $(x,y,z)$:
\be \label{e:ksm:def_r}
   \encadre{ r := \sqrt{ \frac{1}{2} \left(
        x^2 + y^2 + z^2 - a^2 +
        \sqrt{(x^2 + y^2 + z^2 - a^2)^2 + 4 a^2 z^2} \right)} } .
\ee
Actually, $r$ is nothing but the
radial coordinate of Boyer-Lindquist coordinates $(t,r,\th,\ph)$
(cf. Sec.~\ref{s:ker:expr_BL}),
as well as of Kerr coordinates $(v, r, \th, \tph)$ (cf. Sec.~\ref{s:ker:Kerr_coord})
or of the 3+1 Kerr coordinates $(\ti, r, \th,\tph)$ (cf. Sec.~\ref{s:ker:3p1_Kerr_coord}).

\begin{remark}
If $a=0$ (Schwarzschild limit), we get
\be \label{e:ksm:lim_a_zero}
    r = \sqrt{x^2 + y^2 + z^2}, \quad
    H = \frac{m}{r} \quad\mbox{and}\quad
    k_\alpha = \left( 1,\; \frac{x}{r},\; \frac{y}{r},\; \frac{z}{r} \right) .
\ee
\end{remark}

\begin{remark}
For $a \not = 0$, the relations (\ref{e:ksm:lim_a_zero}) hold at first order
in the limit $r \gg a$, or equivalently in the limit $\sqrt{x^2 + y^2 + z^2} \gg a$.
\end{remark}

The Kerr-Schild coordinate system $(\ti,x,y,z)$ actually covers only
the part $r\geq 0$ of the Kerr spacetime as introduced in Sec.~\ref{s:ker:expr_BL}.
For the $r<0$ part, which exists as soon as $a\not=0$,
one has to introduce another patch of Kerr-Schild coordinates, $(\ti,x',y',z')$ say, such
that
\be
    r := - \sqrt{ \frac{1}{2} \left(
        {x'}^2 + {y'}^2 + {z'}^2 - a^2 +
        \sqrt{({x'}^2 + {y'}^2 + {z'}^2 - a^2)^2 + 4 a^2 {z'}^2} \right)} .
\ee


The explicit form of the components
$g_{\mu\nu}$ of the Kerr metric in Kerr-Schild coordinates can be read off by expanding the line
element
\be \label{e:ksm:g_comp_KS}
\encadre{ \begin{array}{ll}
g_{\mu\nu} \, \D x^\mu \, \D x^\nu = & - \D \ti^2 + \D x^2 + \D y^2
        + \D z^2 \\
        &\displaystyle  + \frac{2m r^3}{r^4 + a^2 z^2} \left( \D t
        + \frac{r x + a y}{r^2 + a^2} \, \D x
        + \frac{r y - a x}{r^2 + a^2} \, \D y + \frac{z}{r} \, \D z \right) ^2
\end{array} } ,
\ee
which is obtained by combining Eqs~(\ref{e:ksm:g_KerrSchild_comp}), (\ref{e:ksm:ds_eta}), (\ref{e:ksm:H_Kerr})
and (\ref{e:ksm:k_a_Kerr}).
\begin{remark}
It is clear on \eqref{e:ksm:g_comp_KS} that all metric components in Kerr-Schild
coordinates are regular at the event horizon, which is defined by $r= m + \sqrt{m^2-a^2}$.
This is in sharp contrast with the metric components in the standard
Boyer-Lindquist coordinates [cf. Eq.~(\ref{e:ker:metric_BL})].
\end{remark}

Given expression \eqref{e:ksm:def_r} for $r$, the following relation holds
for $r \not=0$:
\be
\encadre{ \frac{x^2 + y^2}{r^2 + a^2} + \frac{z^2}{r^2} = 1 } .
\ee
This shows that, in the Euclidean space spanned by the $(x,y,z)$ coordinates,
the surfaces of constant $r\not=0$ are
confocal\footnote{In any plane containing the axis of symmetry, the trace of
the ellipsoids are ellipses that share the same foci, located
at the abscissas $\pm a$ along the $z=0$ axis.}
 ellipsoids of revolution. For $r=0$ and $a>0$, Eq.~\eqref{e:ksm:def_r} shows that the surface
of constant $r$ is the disk of radius $a$ centered at the origin in the plane
$z=0$:
\be \label{e:ksm:disk_r0}
    r = 0 \iff \left\{ \begin{array}{l}
        z = 0 \\
        x^2 + y^2 \leq a^2 .
        \end{array} \right.
\ee
\emph{Proof:} starting from Eq.~\eqref{e:ksm:def_r}, we have successively
\bea
    r = 0 & \iff & x^2 + y^2 + z^2 - a^2 =
        - \sqrt{(x^2 + y^2 + z^2 - a^2)^2 + 4 a^2 z^2} \nonumber \\
        & \iff & \begin{cases}
               (x^2 + y^2 + z^2 - a^2)^2 =   (x^2 + y^2 + z^2 - a^2)^2 + 4 a^2 z^2 \\
               x^2 + y^2 + z^2 - a^2 \leq 0
               \end{cases} \nonumber \\
        & \iff &  \begin{cases}
                  4 a^2 z^2 = 0 \\
                  x^2 + y^2 + z^2 \leq a^2,
                 \end{cases}
        \iff
        \begin{cases}
                  z = 0 \\
                  x^2 + y^2 \leq a^2,
                  \end{cases}  \nonumber
\eea
where the last equivalence assumes $a\not=0$. \qed

Since in Kerr-Schild coordinates, $f^{\alpha\beta} = \mathrm{diag}(-1, 1, 1, 1)$,
the components of the null vector $\w{k}$ are immediately
deduced from Eqs.~\eqref{e:ksm:k_up_comp} and \eqref{e:ksm:k_a_Kerr}:
\be \label{e:ksm:k_sharp_comp}
   \encadre{ k^\alpha = \left( -1,\ \frac{r x + a y}{r^2 + a^2},\
        \frac{r y - a x}{r^2 + a^2},\ \frac{z}{r} \right) }.
\ee

Let $\w{\xi} = \wpar_{\ti}$ be the Killing vector of Kerr spacetime
associated with stationarity:
According to the components (\ref{e:ksm:k_a_Kerr})
of $\w{k}$, we have
\be
    \w{g}(\w{\xi}, \w{k}) = \xi^\mu k_\mu = k_{\ti} = 1 > 0 .
\ee
In the regions where $\w{\xi}$ is timelike, i.e. outside the ergoregion,
this implies that the null vector $\w{k}$ is past-directed (assuming
that the time orientation is chosen so that $\w{\xi}$ is future-directed).
Indeed the scalar product of a future-directed timelike vector with a future-directed
null vector is necessarily negative (cf. the lemma in Sec.~\ref{s:fra:time_orientation}).
To get a future-directed null vector, some authors consider $\w{k}' = - \w{k}$
instead of $\w{k}$.

\subsection{Link with Kerr coordinates}

The link between the Kerr-Schild coordinates $(\ti, x, y, z)$ and the 3+1 Kerr
coordinates $(\ti, r, \th, \tph)$ introduced in Sec.~\ref{s:ker:3p1_Kerr_coord} is
\begin{subequations}
\label{e:ksm:K_to_KS}
\bea
    \ti & = & \ti \\
    x & = & (r\cos\tph + a \sin\tph) \sin \th  \label{e:ksm:x_Kerr}\\
    y & = & (r\sin\tph - a \cos\tph) \sin \th  \label{e:ksm:y_Kerr}\\
    z & = & r\cos \th . \label{e:ksm:z_Kerr}
\eea
\end{subequations}
The inverse relation gives birth to Eq.~\eqref{e:ksm:def_r} relating $r$ to
$(x,y,z)$.

\begin{remark}
Equations~\eqref{e:ksm:x_Kerr}-\eqref{e:ksm:y_Kerr} can be combined into a single
relation:
\be
    x + i y = (r - i a) \mathrm{e}^{i\tph} \sin\th .
\ee
\end{remark}

\begin{remark}
For $a=0$, Eqs.~\eqref{e:ksm:x_Kerr}-\eqref{e:ksm:z_Kerr} reduce to the standard
relations between Cartesian and spherical coordinates in Euclidean space.
\end{remark}

Thanks to $z=r\cos\th$ [Eq.~(\ref{e:ksm:z_Kerr})], the scalar field $H$
defined by Eq.~\eqref{e:ksm:H_Kerr} takes a simple expression in terms of
Kerr coordinates:
\be
    H = \frac{mr}{\rho^2} ,
\ee
where $\rho^2 := r^2 + a^2 \cos^2\th$ [Eq.~(\ref{e:ker:def_rho2})].

The link between Kerr-Schild coordinates $(\ti, x, y, z)$ and the Kerr
coordinates $(v, r, \th, \tph)$ introduced in Sec.~\ref{s:ker:Kerr_coord}
is given by $\ti = v - r$ [Eq.~(\ref{e:ker:def_t_tilde})]
and
Eqs.~(\ref{e:ksm:x_Kerr})-(\ref{e:ksm:z_Kerr}).


\subsection{Link with Boyer-Lindquist coordinates}

The link between the Kerr-Schild coordinates $(\ti,x,y,z)$ and the Boyer-Lindquist
coordinates $(t,r,\th,\ph)$
is obtained by combining Eqs.~(\ref{e:ksm:K_to_KS}) with Eqs.~(\ref{e:ker:Kerr_3p1_BL_int}).


\begin{hist}
Kerr-Schild coordinates have been introduced by Roy Kerr\index{Kerr, R.} in his famous 1963 paper
\cite{Kerr63} announcing the discovery of the
metric that bares his name; they have been discussed further by Kerr and Alfred
Schild\index{Schild, A.}
in 1965 \cite{KerrS65}, as well as by
Robert Boyer\index{Boyer, R.H.} and Richard Lindquist\index{Lindquist, R.W.}
in 1967 \cite{BoyerL67}, Brandon Carter\index{Carter, B.} in 1968
\cite{Carte68}
and Stephen Hawking\index{Hawking, S.W.} and George Ellis\index{Ellis, G.F.R} in 1973 \cite{HawkiE73}.
\end{hist}
